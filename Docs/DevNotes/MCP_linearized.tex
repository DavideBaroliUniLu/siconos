 This section is devoted to the implementation and the study  of the algorithm. The interval of integration is $[0,T]$, $T>0$, and a grid $t_{0}=0$, $t_{k3+1}=t_{k}+h$, $k \geq 0$, $t_{N}=T$ is constructed. The approximation of a function $f(\cdot)$ on $[0,T]$ is denoted as $f^{N}(\cdot)$, and is a piecewise constant function, constant on the intervals $[t_{k},t_{k+1})$. We denote $f^{N}(t_{k})$ as $f_{k}$. The time-step is $h>0$. 




\subsection{Time--discretization of the general case~(\ref{eq:quatre}) } 

This fully general case is not yet implemented in Siconos.

Starting from  (\ref{eq:quatre}), let us introduce anew notation, 
\begin{equation}
\begin{array}{l}
\dot{x}(t) = f(x(t),t) + r(t)  \\[2mm]
y(t) = h(t,x(t),\lambda (t)) \\[2mm]
r(t) = g(t,x(t),\lambda (t) ) \\[2mm]
\end{array}
\label{sept-bis-bis}
\end{equation}
where $s(t) \in \RR^m$  and $y(t) \in \RR^m$ are  complementary variables related trough the $Sgn$ multi-valued mapping.   According to the class of systems (\ref{eq:deux}), (\ref{eq:trois}) or (\ref{eq:quatre}) , we are studying, the function $f$ and $g$ are defined by a fully nonlinear framework or by affine functions. We have decided to present the time-discretization in its full generality and specialize the algorithms for each cases in Section~\ref{Sec:Spec}.


Let us now proceed with the time discretization of (\ref{sept-bis-bis}) by a fully implicit scheme : 
\begin{equation}
  \begin{array}{l}
    \label{eq:toto1}
     Mx_{k+1} = x_{k} +h\theta f(x_{k+1},t_{k+1})+h(1-\theta) f(x_k,t_k) + h \theta _r r(t_{k+1})
     + h(1-\theta _r)r(t_k)  \\[2mm]
     y_{k+1} =  h(t+1,x_{k+1},\lambda _{k+1}) \\[2mm]
     r_{k+1} = g(x_{k+\theta},\lambda_{k+1},t+1)\\[2mm]
  \end{array}
\end{equation}
where $\theta = [0,1]$ and $\theta _r \in [0,1]$. As in \cite{acary2008}, we call the problem \eqref{eq:toto1} the ``one--step nonsmooth problem''.

 This time-discretization is slightly more general than a standard implicit Euler scheme. The main discrepancy lies in the choice of a $\theta$-method to integrate the nonlinear term. For $\theta=0$, we retrieve the explicit integration of the smooth and  single valued term $f$. Moreover for $\gamma =0$, the term $g$ is explicitly evaluated. The flexibility in the choice of $\theta$ and $\gamma$ allows the user to improve and control the accuracy, the stability and the numerical damping of the proposed method. For instance, if the smooth dynamics given by $f$ is stiff, or if we have to use big step sizes for practical reasons, the choice of $\theta > 1/2$ offers better stability with the respect to $h$.


\subsection{Newton's linearization}

Due to the fact that  two of the  studied classes of systems that are studied in this paper are affine functions in terms of $f$ and $g$, we propose to solve the "one--step nonsmooth problem'' (\ref{eq:toto1}) by performing an external Newton linearization, which yields a Mixed Linear Complementarity Problems (MLCP).

 \paragraph{Newton's linearization} The first line of the  problem~(\ref{eq:toto1}) can be written under the form of a residue $\mathcal R$ depending only on $x_{k+1}$ and $r_{k+1}$ such that 
\begin{equation}
  \label{eq:NL3}
  \mathcal R (x_{k+1},r _{k+1}) =0
\end{equation}
with $\mathcal R(x,r) = M(x - x_{k}) -h\theta f( x , t_{k+1}) - h(1-\theta)f(x_k,t_k) - h\theta _r r
- h(1-\theta_r)r_k$.
The solution of this system of nonlinear equations is sought as a limit of the sequence $\{ x^{\alpha}_{k+1},r^{\alpha}_{k+1} \}_{\alpha \in \NN}$ such that
 \begin{equation}
   \label{eq:NL7}
   \begin{cases}
     x^{0}_{k+1} = x_k \\ \\
     \mathcal R_L( x^{\alpha+1}_{k+1},r^{\alpha+1}_{k+1}) = \mathcal
     R(x^{\alpha}_{k+1},r^{\alpha}_{k+1})  + \left[ \nabla_{x} \mathcal
     R(x^{\alpha}_{k+1},r^{\alpha}_{k+1})\right] (x^{\alpha+1}_{k+1}-x^{\alpha}_{k+1} ) +
     \left[ \nabla_{r} \mathcal R(x^{\alpha}_{k+1},r^{\alpha}_{k+1})\right] (r^{\alpha+1}_{k+1} - r^{\alpha}_{k+1} ) =0
 \end{cases}
\end{equation}
The residu free is also defined (useful for implementation only):
\[\mathcal R _{free}(x) =  M(x - x_{k}) -h\theta f( x , t_{k+1}) - h(1-\theta)f(x_k,t_k)\]
\[\mathcal R (x,r) = \mathcal R _{free}(x)   - h\theta _r r - h(1-\theta_r)r_k\]
\[ \fbox{$\mathcal R^{\alpha}_{k+1} = \mathcal R (x^{\alpha}_{k+1},r^{\alpha}_{k+1}) = \mathcal R
_{free}(x^{\alpha}_{k+1},r^{\alpha}_{k+1} )  - h\theta _r r^{\alpha}_{k+1} - h(1-\theta_r)r_k$}\]
\[ \mathcal R _{free k+1} ^{\alpha} = \mathcal R
_{free}(x^{\alpha}_{k+1},r^{\alpha}_{k+1} )=\fbox{$M(x^{\alpha}_{k+1} - x_{k}) -h\theta f( x^{\alpha}_{k+1} , t_{k+1}) - h(1-\theta)f(x_k,t_k)$}\]
 
The computation of the Jacobian of $\mathcal R$ with respect to $x$, denoted by $M(x,\lambda)$ leads to 
\begin{equation}
   \label{eq:NL9}
   \begin{array}{l}
    W^{\alpha}_{k+1}= \nabla_{x} \mathcal R (x^{\alpha}_{k+1},r^{\alpha}_{k+1})= M - h  \theta \nabla_{x} f(  x^{\alpha}_{k+1}, t_{k+1} ).\\
 \end{array}
\end{equation}
At each time--step, we have to solve the following linearized problem,
\begin{equation}
   \label{eq:NL10}
    \mathcal R^{\alpha}_{k+1} + W^{\alpha}_{k+1} (x^{\alpha+1}_{k+1} -
    x^{\alpha}_{k+1}) - h \theta _r (r^{\alpha+1}_{k+1} - r^{\alpha}_{k+1} )  =0 ,
\end{equation}
that is

\begin{equation}
   \begin{array}{l}
 h \theta _r  r^{\alpha+1}_{k+1} = r_c + W^{\alpha}_{k+1} x^{\alpha+1}_{k+1}
 .\label{eq:NL11} 
 \end{array}
\end{equation}
with 
\begin{equation}
   \begin{array}{l}
r_c = h \theta _r r^{\alpha}_{k+1} - W^{\alpha}_{k+1} x^{\alpha}_{k+1} + \mathcal R
^{\alpha}_{k+1}=- W^{\alpha}_{k+1} x^{\alpha}_{k+1} + \mathcal R_{free k+1} ^{\alpha} - h(1-\theta_r)r_k\\ \\
\mathcal R ^{\alpha}_{k+1}=M( x^{\alpha}_{k+1} - x_k) -h \theta f(x^{\alpha}_{k+1})-h(1-\theta)f(x_k)
- h \theta _r r^{\alpha}_{k+1} -h(1- \theta _r)r_k

 \end{array}
   \end{equation}
\[x^{\alpha+1}_{k+1} = h(W^{\alpha}_{k+1})^{-1}r^{\alpha+1}_{k+1} +(W^{\alpha}_{k+1})^{-1}(\mathcal
R_{free k+1} ^{\alpha})+x^{\alpha}_{k+1}\]
Using~\ref{eq:NL11}
\[\fbox{$x^{\alpha+1}_{k+1} = h(W^{\alpha}_{k+1})^{-1}r^{\alpha+1}_{k+1} +x_{free}$}\]
with :
\[x_{free}=x^{\alpha}_{k+1}-(W^{\alpha}_{k+1})^{-1}\mathcal R_{freek+1}^{\alpha}\]

The matrix $W$ is clearly non singular for small $h$.
The same operation is performed with the second equation of (\ref{eq:toto1})
\begin{equation}
   \label{eq:NL9}
   \begin{array}{l}
      \mathcal R_y(x,y,\lambda)=y-h(t_{k+1},x,\lambda) =0\\ \\
      \mathcal R_{Ly}(x^{\alpha+1}_{k+1},y^{\alpha+1}_{k+1},\lambda^{\alpha+1}_{k+1}) = \mathcal
      R_{y}(x^{\alpha}_{k+1},y^{\alpha}_{k+1},\lambda^{\alpha}_{k+1}) +
      (y^{\alpha+1}_{k+1}-y^{\alpha}_{k+1})- \\ \qquad \qquad
      C^{\alpha}_{k+1}(x^{\alpha+1}_{k+1}-x^{\alpha}_{k+1}) - D^{\alpha}_{k+1}(\lambda^{\alpha+1}_{k+1}-\lambda^{\alpha}_{k+1})=0\\
      
 \end{array}
\end{equation}

$\mathcal R_y(x^{\alpha+1}_{k+1},y^{\alpha+1}_k+1,\lambda^{\alpha+1}_{k+1})$ leading to the following linearized equation
\begin{equation}
  y^{\alpha+1}_{k+1} =  y^{\alpha}_{k+1}
  -\mathcal R^{\alpha}_{yk+1}+ \\
  C^{\alpha}_{k+1}(x^{\alpha+1}_{k+1}-x^{\alpha}_{k+1}) +
  D^{\alpha}_{k+1}(\lambda^{\alpha+1}_{k+1}-\lambda^{\alpha}_{k+1}). \label{eq:NL11y}
\end{equation}
with,
\begin{equation}
     \begin{array}{l}
  C^{\alpha}_{k+1} = \nabla_xh(t_{k+1}, x^{\alpha}_{k+1},\lambda^{\alpha}_{k+1} ) \\ \\
  D^{\alpha}_{k+1} = \nabla_{\lambda}h(t_{k+1}, x^{\alpha}_{k+1},\lambda^{\alpha}_{k+1})
 \end{array}
\end{equation}
and
\begin{equation}\fbox{$
\mathcal R^{\alpha}_{yk+1} = y^{\alpha}_{k+1} - h(x^{\alpha}_{k+1},\lambda^{\alpha}_{k+1})$}
 \end{equation}

The same operation is performed with the third equation of (\ref{eq:toto1})
\begin{equation}
   \label{eq:NL9}
   \begin{array}{l}
      \mathcal R_r(r,x,\lambda)=r-g(x,\lambda,t_{k+1}) =0\\ \\
      \mathcal R_{L\lambda}(r^{\alpha+1}_{k+1},x^{\alpha+1}_{k+1},\lambda^{\alpha+1}_{k+1}) = \mathcal
      R_{rk+1}^{\alpha} + (r^{\alpha+1}_{k+1} - r^{\alpha}_{k+1}) -
      K^{\alpha}_{k+1}(x^{\alpha+1}_{k+1} - x^{\alpha}_{k+1})- B^{\alpha}_{k+1}(\lambda^{\alpha+1}_{k+1} -
      \lambda^{\alpha}_{k+1})=0\\ \\
      r^{\alpha+1}_{k+1} = r_1 +K^{\alpha}_{k+1}x^{\alpha+1}_{k+1} +B^{\alpha}_{k+1}
      \lambda^{\alpha+1}_{k+1}
            r^{\alpha+1}_{k+1} = \fbox{$g(x ^{\alpha}_{k+1},\lambda ^{\alpha}_{k+1},t_{k+1}) -B^{\alpha}_{k+1}
      \lambda^{\alpha}_{k+1} + B^{\alpha}_{k+1} \lambda^{\alpha+1}_{k+1}-K^{\alpha}_{k+1}
      x^{\alpha}_{k+1} + K^{\alpha}_{k+1} x^{\alpha+1}_{k+1}$}       

 \end{array}
\end{equation}
with,
\begin{equation}
r_1 = g(x^{\alpha}_{k+1},\lambda ^{\alpha}_{k+1},t_{k+1}) - K^{\alpha}_{k+1} x^{\alpha}_{k+1} -
B^{\alpha}_{k+1} \lambda^{\alpha}_{k+1}
\end{equation}
and,
\begin{equation}
     \begin{array}{l}
  K^{\alpha}_{k+1} = \nabla_xg(x^{\alpha}_{k+1},\lambda ^{\alpha}_{k+1},t_{k+1})  \\ \\
  B^{\alpha}_{k+1} = \nabla_{\lambda}g(x^{\alpha}_{k+1},\lambda ^{\alpha}_{k+1},t_{k+1})
 \end{array}
\end{equation}
and the r residue:
\begin{equation}
\mathcal
      R_{rk+1}^{\alpha} = r^{\alpha}_{k+1} - g(x^{\alpha}_{k+1},\lambda ^{\alpha}_{k+1},t_{k+1})
  \end{equation}
Inserting (\ref{eq:NL11}), we get the following linear relation between $x^{\alpha+1}_{k+1}$ and
$\lambda^{\alpha+1}_{k+1}$, 

\begin{equation}
   \begin{array}{l}
    (W^{\alpha}_{k+1})x^{\alpha+1}_{k+1} = h \theta _r \left[g(r^{\alpha}_{k+1},t_{k+1}) +
    B^{\alpha}_{k+1} (\lambda^{\alpha+1}_{k+1} - \lambda^{\alpha}_{k+1})+K^{\alpha}_{k+1}
    (x^{\alpha+1}_{k+1} - x^{\alpha}_{k+1}) \right ] -r_c \\ \\
     x^{\alpha+1}_{k+1} = h\theta _r(W^{\alpha}_{k+1} )^{-1}\left[g(x^{\alpha}_{k+1},\lambda^{\alpha}_{k+1},t_{k+1}) +
    B^{\alpha}_{k+1} (\lambda^{\alpha+1}_{k+1} - \lambda^{\alpha}_{k+1})+K^{\alpha}_{k+1}
    (x^{\alpha+1}_{k+1} - x^{\alpha}_{k+1}) \right ] - \\(W^{\alpha}_{k+1} )^{-1}(- W^{\alpha}_{k+1} x^{\alpha}_{k+1} + \mathcal R_{free k+1} ^{\alpha} - h(1-\theta_r)r_k)\\ \\
    x^{\alpha+1}_{k+1} =  x_p + h \theta _r (W^{\alpha}_{k+1})^{-1}
    (B^{\alpha}_{k+1} \lambda^{\alpha+1}_{k+1}+ K^{\alpha}_{k+1} x^{\alpha+1}_{k+1}) \\ \\
    (I-h \theta _r (W^{\alpha}_{k+1})^{-1}K^{\alpha}_{k+1})x^{\alpha+1}_{k+1}=x_p + h \theta _r (W^{\alpha}_{k+1})^{-1}    B^{\alpha}_{k+1} \lambda^{\alpha+1}_{k+1}

   \end{array}
\end{equation}
We now define:
\[\tilde K^{\alpha}_{k+1}=(I-h \theta _r (W^{\alpha}_{k+1})\]
with $\theta _r =1$:
\[(W^{\alpha}_{k+1} )x^{\alpha+1}_{k+1}= hr^{\alpha+1}_{k+1}- \mathcal R_{free k+1} ^{\alpha}+W^{\alpha}_{k+1}x^{\alpha}_{k+1}\]
\[x^{\alpha+1}_{k+1}= h( W^{\alpha}_{k+1})^{-1}r^{\alpha+1}_{k+1}-
( W^{\alpha}_{k+1})^{-1} \mathcal R_{free k+1} ^{\alpha}+x^{\alpha}_{k+1}\]
\[x^{\alpha+1}_{k+1}= h( W^{\alpha}_{k+1})^{-1}r^{\alpha+1}_{k+1}+x_{free}\]
with, using \ref{}
\begin{equation}
x_p-x^{\alpha}_{k+1}=h(
W^{\alpha}_{k+1})^{-1}(g(x^{\alpha}_{k+1},\lambda^{\alpha}_{k+1},t_{k+1})-B^{\alpha}_{k+1}
\lambda^{\alpha}_{k+1}-K^{\alpha}_{k+1} x^{\alpha}_{k}))+\tilde x_{free}
\end{equation}
\[    \tilde x_{free}= -( W^{\alpha}_{k+1})^{-1} \mathcal R _{free k+1} ^{\alpha} \]
      \[x_{free} = \tilde x_{free} + x^{\alpha}_{k+1}=\fbox{$- W^{-1}R_{free k+1} ^{\alpha} + x^{\alpha}_{k+1}$}\]
\[ \fbox{$x_p  = x_{free} + h ( W^{\alpha}_{k+1})^{-1}( g(x ^{\alpha}_{k+1},\lambda ^{\alpha}_{k+1},t_{k+1}) -
      B^{\alpha}_{k+1} \lambda^{\alpha}_{k+1}-K^{\alpha}_{k+1} x^{\alpha}_{k+1} )$} \]


Inserting (\ref{eq:NL11y}), we get the following linear relation between $y^{\alpha+1}_{k+1}$ and $\lambda^{\alpha+1}_{k+1}$, 
\begin{equation}
   \begin{array}{l}
 y^{\alpha+1}_{k+1} = y_p + \left[ h  C^{\alpha}_{k+1} (\tilde K^{\alpha}_{k+1})^{-1}( W^{\alpha}_{k+1})^{-1}  B^{\alpha}_{k+1} + D^{\alpha}_{k+1} \right]\lambda^{\alpha+1}_{k+1}
   \end{array}
\end{equation}

with 
\begin{equation}\fbox{$
y_p = y^{\alpha}_{k+1} -\mathcal R^{\alpha}_{yk+1} + C^{\alpha}_{k+1}((\tilde K^{\alpha}_{k+1})^{-1}x_p -x^{\alpha}_{k+1}) -
D^{\alpha}_{k+1} \lambda^{\alpha}_{k+1} $}
\end{equation}

\paragraph{Mixed linear complementarity problem (MLCP)}To summarize, the problem to be solved in each Newton iteration is:\\{
  \begin{minipage}[l]{1.0\linewidth}
    \begin{equation}
      \begin{cases}
      \begin{array}[l]{l}
        y^{\alpha+1}_{k+1} =   W_{mlcpk+1}^{\alpha}  \lambda^{\alpha+1}_{k+1} + b^{\alpha}_{k+1}
        \\ \\
        -y^{\alpha+1}_{k+1} \in N_{[l,u]}(\lambda^{\alpha+1}_{k+1} ). 
      \end{array}
      \label{eq:NL14}
      \end{cases}
    \end{equation}
  \end{minipage}
}
with $W_{mlcpk+1}\in \RR^{m\times m}$ and $b\in\RR^{m}$ defined by
\begin{equation}
  \label{eq:NL15}
 \begin{array}[l]{l}
   W_{mlcpk+1}^{\alpha} = h  C^{\alpha}_{k+1} (\tilde K^{\alpha}_{k+1})^{-1} (W^{\alpha}_{k+1})^{-1}  B^{\alpha}_{k+1} + D^{\alpha}_{k+1} \\
   b^{\alpha}_{k+1} = y_p
\end{array}
\end{equation}

The problem~(\ref{eq:NL14}) is equivalent to a Mixed Linear Complementarity Problem (MLCP) which can be solved under suitable assumptions by many linear complementarity solvers such as pivoting techniques, interior point techniques and splitting/projection strategies. The  reformulation into a standard MLCP follows the same line as for the MCP in the previous section. One obtains,
    \begin{equation}
      \begin{array}[l]{l}
        y^{\alpha+1}_{k+1} =   - W^{\alpha}_{k+1}  \lambda^{\alpha+1}_{k+1} + b^{\alpha}_{k+1}
        \\ \\
        (y^{\alpha+1}_{k+1})_i  = 0 \qquad \textrm{ for } i \in \{ 1..n\}\\[2mm]
        0 \leq  (\lambda^{\alpha+1}_{k+1})_i\perp (y^{\alpha+1}_{k+1})_i \geq 0 \qquad \textrm{ for } i \in \{ n..n+m\}\\
      \end{array}
      \label{eq:MLCP1} 
    \end{equation}

\paragraph{MLCP solvers.} As for MCP, there exists numerous methods to numerically solve MLCP. In the worst case when the matrix $W^{\alpha+1}_{k+1}$ has no special properties, the MCLP can be always solved by enumerative solvers for which various implementation can be found. With positivity, P-matrix  or co-positivity properties, standard methods for LCP\cite{Cottle.Pang.ea1992} can be straightforwardly extended. Among these methods, we can cite the family of projection/splitting methods, interior point methods and semi-smooth Newton methods (see \cite{acary2008} for an overview of various types of methods.).

\subsection{The special cases of the  affine  systems~ (\ref{eq:deux}) and ~ (\ref{eq:trois}) }
\label{Sec:Spec}


In this section, we specify the time--discretization of the fully nonlinear case for the two other classes of systems~(\ref{eq:deux}) and ~(\ref{eq:trois}) and for particular value of $\theta$ and $\gamma$. 



%\subsection{Algorithms} 

%We propose in this section two algorithms to sum-up the numerical implementation of the implicit
%Euler time--stepping scheme.
%The Algorithm~\ref{Algo:EulerSliding-MCP} describes the implementation
%with a generic MCP solvers and
%The Algorithm~\ref{Algo:EulerSliding-MCLP} describes the  numerical implementation of the algorithm with an external Newton linearization and a MCLP solver.
%\begin{algorithm}[htbp]
%   \begin{algorithmic}
% { \sf 
%    \REQUIRE System definition: $\sf f, g, h$ 
%    \REQUIRE $\sf x(0)$ the initial condition
%    \REQUIRE $\sf t_0, T$ time--integration interval
%    \REQUIRE $\sf h$ time--step 
%    \REQUIRE $\sf \theta, \gamma$ numerical integration parameters
%    \ENSURE  $\sf (\{ x_k\}, \{ s_k\},\{y_k \}), k \in\{1,2, \ldots \}$ 
%    \STATE $ \sf $
%    \STATE $\sf k \leftarrow 0;\quad  x_{0} \leftarrow x(0);\quad \sf y_{0} \leftarrow y(0) =h(x(0));\quad \sf tau_{0} \leftarrow 0 $  
%    \STATE //\textit{ Time integration loop}
%     \WHILE {$\sf t_k < T$} 
%         \STATE Solve the MCP~(\ref{eq:MCP5}) for $\sf x_{k+1}, s_{k+1}, y_{k+1}$ with $\sf F, l$ and $\sf u$ given by ~(\ref{eq:MCP3})  and the Jacobian $\nabla_z F(z)$ given by (\ref{eq:JacMCLP}).
%         \STATE //\textit{Update}
%         \STATE $\sf x_{k} \leftarrow x_{k+1};\quad \sf s_{k} \leftarrow s_{k+1} ;\quad \sf y_{k} \leftarrow y_{k+1} $
%         \STATE //\textit{time iteration}
%         \STATE $\sf t_{k} \leftarrow t_{k+1};\quad \sf k \leftarrow k+1$
%      \ENDWHILE
%     \STATE $ \sf$}
%   \end{algorithmic}
%   \caption{Implicit Euler time-discretization with a generic MCP solver}  
% \label{Algo:EulerSliding-MCP}
%\end{algorithm}
%\begin{algorithm}[htbp]
%   \begin{algorithmic}
% { \sf 
%    \REQUIRE System definition: $\sf f, g, h$ 
%    \REQUIRE $\sf x(0)$ the initial condition
%    \REQUIRE $\sf t_0, T$ time--integration interval
%    \REQUIRE $\sf h$ time--step
%    \REQUIRE $\sf \theta, \gamma$ numerical integration parameters 
%    \REQUIRE $\sf \varepsilon $ Newton's method tolerance
%    \ENSURE  $\sf (\{ x_k\}, \{ \lambda_k\},\{y_k \}), k \in\{1,2, \ldots \}$ 
%    \STATE $ \sf $
%    \STATE $\sf k \leftarrow 0;\quad  x_{0} \leftarrow x(0);\quad \sf y_{0} \leftarrow y(0) =h(x(0));\quad \sf tau_{0} \leftarrow 0 $  
%    \STATE $ $
%    \STATE //\textit{ Time integration loop}
%     \WHILE {$\sf t_k < T$}
%         \STATE $\sf \alpha \leftarrow 0;\quad\sf x^0_{k+1} \leftarrow x^0_{k};\quad s^0_{k+1} \leftarrow s^0_{k};\quad \sf y^0_{k+1} \leftarrow y^0_{k} $
%        \STATE //\textit{Newton's loop}
%          \WHILE {$\sf  \| \mathcal R(x^{\alpha}_{k+1},\lambda^{\alpha}_{k+1} )\|+\| \mathcal R_y(x^{\alpha}_{k+1},\lambda^{\alpha}_{k+1} )\|+\| \mathcal R_r(x^{\alpha}_{k+1},s^{\alpha}_{k+1} )\| > \varepsilon$}  
%         \STATE  $\sf M^{-1}(x^{\alpha}_{k+1},s^{\alpha}_{k+1} ) \leftarrow (I - h \theta\nabla_x f(x^{\alpha}_{k+\theta},t_{k+1}) - h \gamma \nabla_x g(t_{k+\theta},x^{\alpha}_{k+\gamma},\lambda^{\alpha}_{k+1})  )^{-1} $.
%         \STATE  $\sf W^{\alpha+1}_{k+1} \leftarrow D^{\alpha}_{k+1} -hC^{\alpha}_{k+1}(M^{\alpha}_{k+1})^{-1}B^{\alpha}_{k+1}$
%           \STATE  $\sf b^{\alpha+1}_{k+1} \leftarrow y^{\alpha}_{k+1}-\mathcal R_{yk+1}^{\alpha}+
%  \left[hC^{\alpha}_{k+1}(M^{\alpha}_{k+1})^{-1}B^{\alpha}_{k+1} - D^{\alpha}_{k+1}\right]
%  \lambda^{\alpha}_{k+1} - C^{\alpha}_{k+1}(M^{\alpha}_{k+1})^{-1} \mathcal R_{k+1}^{\alpha} $
%           \STATE $ $
%           \STATE Solve the MLCP (\ref{eq:MLCP1}) for $\sf y^{\alpha+1}_{k+1}, s^{\alpha+1}_{k+1}$
%           \STATE $ $
%          \STATE $\sf \lambda^{\alpha}_{k+1} \leftarrow \lambda^{\alpha+1}_{k+1};\quad  y^{\alpha}_{k+1} \leftarrow y^{\alpha+1}_{k+1} $
%         \STATE $\sf x^{\alpha}_{k+1}\leftarrow x^{\alpha}_{k+1} - xp^{\alpha}_{k+1}-h(M^{\alpha}_{k+1})^{-1}B^{\alpha}_{k+1}\lambda^{\alpha}_{k+1}$
%         \STATE $\sf \alpha \leftarrow \alpha+1$         
%         \ENDWHILE
%         \STATE //\textit{Update}
%         \STATE $\sf x_{k+1}\leftarrow x^{\alpha}_{k+1};\quad \sf \lambda_{k+1} \leftarrow \lambda^{\alpha+1}_{k+1} ;\quad \sf y_{k+1} \leftarrow y^{\alpha+1}_{k+1} $
%         \STATE //\textit{time iteration}
%         \STATE $\sf t_{k} \leftarrow t_{k+1};\quad \sf k \leftarrow k+1$
%      \ENDWHILE
%     \STATE $ \sf$}
%   \end{algorithmic}
%   \caption{Implicit Euler time-discretization with an external Newton loop and a MLCP solver }  
% \label{Algo:EulerSliding-MCLP}
%\end{algorithm}

%In the case of the system~(\ref{eq:deux}) with a affine function $f$ or $\theta =0$, the the MLCP matrix $W$ can be computed before the beginning of the time loop saving a lot of computing effort.  In the case of the system (\ref{eq:trois}) with $\theta=\gamma=0$, the MLCP matrix $W$ can be computed before the beginning of the Newton loop.
\clearpage


%%% Local Variables: 
%%% mode: latex
%%% TeX-master: "EulerSliding"
%%% End: 
