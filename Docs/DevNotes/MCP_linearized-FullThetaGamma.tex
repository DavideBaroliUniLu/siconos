\section{Newton's linearization of~(\ref{first-DS2}) with  (\ref{eq:toto1-ter}) } 

  \begin{equation}
    \begin{array}{l}
      \label{eq:full-toto1-ter}
      M x_{k+1} = M x_{k} +h \theta f(x_{k+1},t_{k+1}) +h(1-\theta)f(x_{k},t_{k}) + h r_{k+\gamma} \\[2mm]
      y_{k+\gamma} =  h(t_{k+\gamma},x_{k+\gamma},\lambda _{k+\gamma}) \\[2mm]
      r_{k+\gamma} = g(\lambda_{k+\gamma},t_{k+\gamma})\\[2mm]
    \end{array}
\end{equation}

 \paragraph{Newton's linearization of the first line of~(\ref{eq:full-toto1-ter})} The first line of the  problem~(\ref{eq:full-toto1}) can be written under the form of a residue $\mathcal R$ depending only on $x_{k+1}$ and $r_{k+\gamma}$ such that 
\begin{equation}
  \label{eq:full-NL3}
  \mathcal R (x_{k+1},r _{k+\gamma}) =0
\end{equation}
with $\mathcal R(x,r) = M(x - x_{k}) -h\theta f( x , t_{k+1}) - h(1-\theta)f(x_k,t_k) - h r $
The solution of this system of nonlinear equations is sought as a limit of the sequence $\{ x^{\alpha}_{k+1},r^{\alpha}_{k+\gamma} \}_{\alpha \in \NN}$ such that
 \begin{equation}
   \label{eq:full-NL7}
   \begin{cases}
     x^{0}_{k+1} = x_k \\ \\
     \mathcal R_L( x^{\alpha+1}_{k+1},r^{\alpha+1}_{k+1}) = \mathcal
     R(x^{\alpha}_{k+1},r^{\alpha}_{k+1})  + \left[ \nabla_{x} \mathcal
     R(x^{\alpha}_{k+1},r^{\alpha}_{k+1})\right] (x^{\alpha+1}_{k+1}-x^{\alpha}_{k+1} ) +
     \left[ \nabla_{r} \mathcal R(x^{\alpha}_{k+1},r^{\alpha}_{k+\gamma})\right] (r^{\alpha+1}_{k+\gamma} - r^{\alpha}_{k+\gamma} ) =0
 \end{cases}
\end{equation}
\begin{ndrva}
  What about $r^0_{k+\gamma}$ ?
\end{ndrva}

The residu free is also defined (useful for implementation only):
\[\mathcal R _{free}(x) \stackrel{\Delta}{=}  M(x - x_{k}) -h\theta f( x , t_{k+1}) - h(1-\theta)f(x_k,t_k),\]
which yields
\[\mathcal R (x,r) = \mathcal R _{free}(x)   - h r .\]

\begin{equation}
  \mathcal R (x^{\alpha}_{k+1},r^{\alpha}_{k+\gamma}) = \fbox{$\mathcal R^{\alpha}_{k+1} \stackrel{\Delta}{=}  \mathcal R_{free}(x^{\alpha}_{k+1} )  - h r^{\alpha}_{k+\gamma}$}\label{eq:full-rfree-1}
\end{equation}

\[  \mathcal R
_{free}(x^{\alpha}_{k+1} )=\fbox{$ \mathcal R _{free k+1} ^{\alpha} \stackrel{\Delta}{=}  M(x^{\alpha}_{k+1} - x_{k}) -h\theta f( x^{\alpha}_{k+1} , t_{k+1}) - h(1-\theta)f(x_k,t_k)$}\]
 
The computation of the Jacobian of $\mathcal R$ with respect to $x$, denoted by $   W^{\alpha}_{k+1}$ leads to 
\begin{equation}
   \label{eq:full-NL9}
   \begin{array}{l}
    W^{\alpha}_{k+1} \stackrel{\Delta}{=} \nabla_{x} \mathcal R (x^{\alpha}_{k+1},r^{\alpha}_{k+1})= M - h  \theta \nabla_{x} f(  x^{\alpha}_{k+1}, t_{k+1} ).\\
 \end{array}
\end{equation}
At each time--step, we have to solve the following linearized problem,
\begin{equation}
   \label{eq:full-NL10}
    \mathcal R^{\alpha}_{k+1} + W^{\alpha}_{k+1} (x^{\alpha+1}_{k+1} -
    x^{\alpha}_{k+1}) - h  (r^{\alpha+1}_{k+\gamma} - r^{\alpha}_{k+\gamma} )  =0 ,
\end{equation}
By using (\ref{eq:full-rfree-1}), we get
\begin{equation}
  \label{eq:full-rfree-2}
  \mathcal R _{free}(x^{\alpha}_{k+1})  - h  r^{\alpha+1}_{k+\gamma}   + W^{\alpha}_{k+1} (x^{\alpha+1}_{k+1} -
    x^{\alpha}_{k+1})  =0 
\end{equation}

%\fbox
{
  \begin{equation}
    \boxed{ x^{\alpha+1}_{k+1} = h(W^{\alpha}_{k+1})^{-1}r^{\alpha+1}_{k+1} +x^\alpha_{free}}
  \end{equation}
}
with :
\begin{equation}
  \boxed{x^\alpha_{free}\stackrel{\Delta}{=}x^{\alpha}_{k+1}-(W^{\alpha}_{k+1})^{-1}\mathcal R_{freek+1}^{\alpha} \label{eq:full-rfree-12}}
\end{equation}

The matrix $W$ is clearly non singular for small $h$.


 \paragraph{Newton's linearization of the second  line of~(\ref{eq:full-toto1})}
The same operation is performed with the second equation of (\ref{eq:full-toto1})
\begin{equation}
  \begin{array}{l}
    \mathcal R_y(x,y,\lambda)=y-h(t_{k+\gamma},\gamma x + (1-\gamma) x_k ,\lambda) =0\\ \\
  \end{array}
\end{equation}
which is linearized as
\begin{equation}
  \label{eq:full-NL9}
  \begin{array}{l}
    \mathcal R_{Ly}(x^{\alpha+1}_{k+1},y^{\alpha+1}_{k+\gamma},\lambda^{\alpha+1}_{k+\gamma}) = \mathcal
    R_{y}(x^{\alpha}_{k+1},y^{\alpha}_{k+\gamma},\lambda^{\alpha}_{k+\gamma}) +
    (y^{\alpha+1}_{k+\gamma}-y^{\alpha}_{k+\gamma})- \\[2mm] \qquad  \qquad \qquad \qquad  \qquad \qquad
    \gamma C^{\alpha}_{k+1}(x^{\alpha+1}_{k+1}-x^{\alpha}_{k+1}) - D^{\alpha}_{k+\gamma}(\lambda^{\alpha+1}_{k+\gamma}-\lambda^{\alpha}_{k+\gamma})=0
  \end{array}
\end{equation}

This leads to the following linear equation
\begin{equation}
  \boxed{y^{\alpha+1}_{k+\gamma} =  y^{\alpha}_{k+\gamma}
  -\mathcal R^{\alpha}_{y,k+1}+ \\
  C^{\alpha}_{k+1}(x^{\alpha+1}_{k+1}-x^{\alpha}_{k+1}) +
  D^{\alpha}_{k+\gamma}(\lambda^{\alpha+1}_{k+\gamma}-\lambda^{\alpha}_{k+\gamma})}. \label{eq:full-NL11y}
\end{equation}
with,
\begin{equation}
     \begin{array}{l}
  C^{\alpha}_{k+1} = \nabla_xh(t_{k+1}, x^{\alpha}_{k+1},\lambda^{\alpha}_{k+\gamma} ) \\ \\
  D^{\alpha}_{k+\gamma} = \nabla_{\lambda}h(t_{k+1}, x^{\alpha}_{k+1},\lambda^{\alpha}_{k+\gamma})
 \end{array}
\end{equation}
and
\begin{equation}\fbox{$
\mathcal R^{\alpha}_{yk+1} \stackrel{\Delta}{=} y^{\alpha}_{k+\gamma} - h(x^{\alpha}_{k+1},\lambda^{\alpha}_{k+\gamma})$}
 \end{equation}
 \paragraph{Newton's linearization of the third  line of~(\ref{eq:full-toto1})}
The same operation is performed with the third equation of (\ref{eq:full-toto1})
\begin{equation}
  \begin{array}{l}
    \mathcal R_r(r,\lambda)=r-g(\lambda,t_{k+1}) =0\\ \\  \end{array}
\end{equation}
which is linearized as
\begin{equation}
  \label{eq:full-NL9}
  \begin{array}{l}
      \mathcal R_{L\lambda}(r^{\alpha+1}_{k+\gamma},\lambda^{\alpha+1}_{k+\gamma}) = \mathcal
      R_{rk+\gamma}^{\alpha} + (r^{\alpha+1}_{k+\gamma} - r^{\alpha}_{k+\gamma}) - B^{\alpha}_{k+\gamma}(\lambda^{\alpha+1}_{k+\gamma} -
      \lambda^{\alpha}_{k+\gamma})=0
    \end{array}
  \end{equation}
\begin{equation}
  \label{eq:full-rrL}
  \begin{array}{l}
    \boxed{r^{\alpha+1}_{k+\gamma} = g(\lambda ^{\alpha}_{k+\gamma},t_{k+\gamma}) -B^{\alpha}_{k+\gamma}
      \lambda^{\alpha}_{k+\gamma} + B^{\alpha}_{k+\gamma} \lambda^{\alpha+1}_{k+\gamma}}       
  \end{array}
\end{equation}
with,
\begin{equation}
     \begin{array}{l}
  B^{\alpha}_{k+\gamma} = \nabla_{\lambda}g(\lambda ^{\alpha}_{k+\gamma},t_{k+\gamma})
 \end{array}
\end{equation}
and the  residue for $r$:
\begin{equation}
\boxed{\mathcal
      R_{rk+\gamma}^{\alpha} = r^{\alpha}_{k+\gamma} - g(\lambda ^{\alpha}_{k+\gamma},t_{k+\gamma})}
  \end{equation}
  \begin{ndrva}
    stop here

  \end{ndrva}

\paragraph{Reduction to a linear relation between  $x^{\alpha+1}_{k+1}$ and
$\lambda^{\alpha+1}_{k+\gamma}$}

Inserting (\ref{eq:full-rrL}) into~(\ref{eq:full-rfree-12}), we get the following linear relation between $x^{\alpha+1}_{k+1}$ and
$\lambda^{\alpha+1}_{k+1}$, 

\begin{equation}
   \begin{array}{l}
     x^{\alpha+1}_{k+1} = h\gamma(W^{\alpha}_{k+1} )^{-1}\left[g(x^{\alpha}_{k+1},\lambda^{\alpha}_{k+1},t_{k+1}) +
    B^{\alpha}_{k+1} (\lambda^{\alpha+1}_{k+1} - \lambda^{\alpha}_{k+1})+K^{\alpha}_{k+1}
    (x^{\alpha+1}_{k+1} - x^{\alpha}_{k+1}) \right ] +x^\alpha_{free}
\end{array}
\end{equation}
that is 
\begin{equation}
  \begin{array}{l}
    (I-h \gamma (W^{\alpha}_{k+1})^{-1}K^{\alpha}_{k+1})x^{\alpha+1}_{k+1}=x_p + h \gamma (W^{\alpha}_{k+1})^{-1}    B^{\alpha}_{k+1} \lambda^{\alpha+1}_{k+1}
   \end{array}
\end{equation}
with 
\begin{equation}
  \boxed{x_p \stackrel{\Delta}{=}  h\gamma(W^{\alpha}_{k+1} )^{-1}\left[g(x^{\alpha}_{k+1},\lambda^{\alpha}_{k+1},t_{k+1}) +
    -B^{\alpha}_{k+1} (\lambda^{\alpha}_{k+1})-K^{\alpha}_{k+1} (x^{\alpha}_{k+1}) \right ] +x^\alpha_{free}}
\end{equation}



Let us  define the new matrix
\begin{equation}
\hat K^{\alpha}_{k+1}=(I-h \gamma (W^{\alpha}_{k+1})K^{\alpha}_{k+1}).
\label{eq:full-hatW}
\end{equation}
We get the linear relation
\begin{equation}
  \label{eq:full-rfree-13}
  \begin{array}{l}
 \boxed{   x^{\alpha+1}_{k+1}\stackrel{\Delta}{=} \hat K^{\alpha,-1}_{k+1} x_p + \hat K^{\alpha,-1}_{k+1} \left[ h \gamma (W^{\alpha}_{k+1})^{-1}    B^{\alpha}_{k+1} \lambda^{\alpha+1}_{k+1}\right]}
   \end{array}
\end{equation}


\begin{ndrva}
  Olivier: Could you confirm the definition of $x_q$ ?
\end{ndrva}




\paragraph{Reduction to a linear relation between  $y^{\alpha+1}_{k+\gamma}$ and
$\lambda^{\alpha+1}_{k+\gamma}$}

Inserting (\ref{eq:full-rfree-13}) into (\ref{eq:full-NL11y}), we get the following linear relation between $y^{\alpha+1}_{k+1}$ and $\lambda^{\alpha+1}_{k+1}$, 
\begin{equation}
   \begin{array}{l}
 y^{\alpha+1}_{k+1} = y_p + \left[ h  C^{\alpha}_{k+1} (\tilde K^{\alpha}_{k+1})^{-1}( W^{\alpha}_{k+1})^{-1}  B^{\alpha}_{k+1} + D^{\alpha}_{k+1} \right]\lambda^{\alpha+1}_{k+1}
   \end{array}
\end{equation}
with 
\begin{equation}\boxed{
y_p = y^{\alpha}_{k+1} -\mathcal R^{\alpha}_{yk+1} + C^{\alpha}_{k+1}(x_q) -
D^{\alpha}_{k+1} \lambda^{\alpha}_{k+1} }
\end{equation}
\textcolor{red}{
  \begin{equation}
   \boxed{ x_q=(\tilde K^{\alpha}_{k+1})^{-1}x_p -x^{\alpha}_{k+1})\label{eq:full-xqq}}
  \end{equation}
}








\clearpage


%%% Local Variables: 
%%% mode: latex
%%% TeX-master: "DevNotes"
%%% End: 
