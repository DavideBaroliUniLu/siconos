\subsection{Gradiant computaion, case oif NewtonEuler with quaternion}

In the section, $q$ is the quaternion of the dynamical system.

\begin{figure}[h]
  \centering
   
  \input{./Figures/NewtonEulerImpact.pstex_t}
  
  \caption{Impact of one DS.}
  \label{figCase}
\end{figure}

$\nabla _q h$ consist in computing $P_c(q+\delta q)-P_c(q)$.
\[GP(q)=qG_0P_0~^cq\]
\[GP(q+\delta q)=(q+\delta q)G_0P_0~^c(q+\delta q)\]
\[=(q+\delta q)~^cqGP(q)q~^c(q+\delta q)\]
\[=(1,0,0,0)+\delta q~^cq)GP(q)(q~^cq+q~^c\delta q)\]
\[=GP(q)+\delta q~^cqGP(q) + GP(q)q~^c\delta q+0(\delta q)^2\]
So, because G is independant of $q$:
\[P(q+\delta q)-P(q)=qGP(q+\delta q)-GP(q)=\delta q~^cqGP(q) + GP(q)q~^c\delta q+0(\delta q)^2\]
For the directional derivation, we chose $\delta q = \epsilon * (1,0,0,0)$
\[\lim_{\epsilon \to 0}\frac{P(q+\delta q)-P(q)}{\epsilon}=~^cqGP(q) + GP(q)q\]
For the directional derivation, we chose $\delta q = \epsilon * (0,1,0,0)=\epsilon * e_i$
\[\lim_{\epsilon \to 0}\frac{P(q+\delta q)-P(q)}{\epsilon}=e_i~^cqGP(q) - GP(q)qe_i\]
Application to the NewtonEulerRImpact:
\[H:\mathbb{R}^7 \to \mathbb{R}\]
\[\nabla _q H \in \mathcal{M}^{1,7}\]
\[\nabla _q H =\left(\begin{array}{c} N_x\\N_y\\N_z\\
(~^cqGP(q) + GP(q)q).N\\
(e_2~^cqGP(q) - GP(q)qe_2).N\\
(e_3~^cqGP(q) - GP(q)qe_3).N\\
(e_4~^cqGP(q) - GP(q)qe_4).N\\
\end{array}\right)\]
\subsection{Ball case}
It is the case where $GP=-N$:
for $e2$:
\[(0,1,0,0).(q_0,-\underline p).(0,-N)=\]
\[\left(\left(\begin{array}{c}1\\0\\0\end{array}\right).\underline p,\left(\begin{array}{c}q_0\\0\\0\end{array}\right) -\left(\begin{array}{c}1\\0\\0\end{array}\right)*\underline p \right).(0,-N)=\]
\[\left(?, -\underline p_x~N-\left(\left(\begin{array}{c}q_0\\0\\0\end{array}\right)- \left(\begin{array}{c}1\\0\\0\end{array}\right)*\underline p \right)*N\right)=\]
and:
\[(0,-N).(q_0,\underline p).(0,1,0,0)=\]
\[(N.\underline p,-q_0N-N*\underline p).(0,1,0,0)=\]
\[\left(?,(N.\underline p)\left(\begin{array}{c}1\\0\\0\end{array}\right) + \left(\begin{array}{c}1\\0\\0\end{array}\right)*(q_0N+N*\underline p)\right)=\]
\[\left(?,(N.\underline p)\left(\begin{array}{c}1\\0\\0\end{array}\right)+q_0 \left(\begin{array}{c}1\\0\\0\end{array}\right)*N+\left(\begin{array}{c}1\\0\\0\end{array}\right)*(N*\underline p)\right)\]
sub then and get the resulting vector.N:
\[\left[ -\underline p_x~N -N.\underline p~\left(\begin{array}{c}1\\0\\0\end{array}\right)+()*N-\left(\begin{array}{c}1\\0\\0\end{array}\right)*(N*\underline p)\right].N=\]
\[-\underline p_x-N_xN.\underline p+0-(\left(\begin{array}{c}1\\0\\0\end{array}\right)*(N*\underline p)).N=\]
  using $a*(b*c)=b(a.c)-c(a.b)$ leads to
  \[-q_1-N_xN.\underline p-(q_1~N-N_x~\underline p).N=\]
\[-q_1-N_xN.\underline p-q_1+N_xN.\underline p=-2q_1\]
for $e1=(1,0,0,0)$:
\[(q_0,-\underline p).(0,-N)=(?,-q_0N+\underline p*N)\]
\[(0,-N).(q_0,\underline p)=(?,-q_0N-\underline p*N)\]
So
\[\nabla _q H =\left(\begin{array}{c} N_x\\N_y\\N_z\\
-2q_0\\
-2q_1\\
-2q_2\\
-2q_3\\
\end{array}\right)\]

The question is $(\nabla _q H)^t \delta q$ where $\delta q $ is tengantial to the sphere:
\[\delta q=(cos((x+\delta x)/2),sin((x+\delta x)/2)(p \delta p))-(cos(x/2),sin(x/2)p)\]
\[\delta q=(cos(x/2)cos(\delta x/2)-sin(x/2)sin(\delta x/2),(sin(x/2)cos(\delta x/2)+cos(x/2)sin(\delta x/2)(p+\delta p)-(cos(x/2),sin(x/2)p)\]
\[\delta q=(\frac{\delta x sin(x/2)}{2}+0(\delta x)^2,\delta x/2cos(x/2)p+\delta p sin(x/2) + 0(\delta x)^2)\]
\[\delta q=(-\delta x/2sin(x/2)+0(\delta x)^2,\delta x/2cos(x/2)p+\delta p sin(x/2)+0(\delta )^2)\]
and
\[(\nabla _q H)^t \delta q = -\delta x cos(x/2)sin(x/2)+\delta x cos(x/2)sin(x/2)+2\delta p .p sin(x/2) +O(\delta)^2\]
Moreover $\delta p .p =0$
\[(\nabla _q H)^t \delta q = O(\delta)^2\]
That shows that this method is a second order aproximation.

\subsection{Case FC3D: using the local frame}

\[\left(\begin{array}{c}m \dot V\\I \dot \omega + \omega I \omega \end{array}\right)= \left(\begin{array}{c}Fect+R\\Mext + R*PG \end{array}\right)\]
  with * vectoriel product, $R$ reaction in the globla frame. $P$ the point of contact.
  $r$ is the reaction in the local frame.  $M^t r=R$ with:
  \[M^t=\left(\begin{array}{ccc} nx&t_1x&t_2x \\ny&t_1y&t_2y\\nz&t_1z&t_2z \end{array}\right)\]
  we have :
  \[\left(\begin{array}{c}R\\R*PG\end{array}\right)=\left(\begin{array}{ccc} 1&0&0\\0&1&0\\0&0&1\\
      0&-PG_z&PG_y\\PG_z&0&-PG_x\\-PG_y&PG_X&0\end{array}\right).R:=N^tR=N^tM^tr\]
      we want:
      
\[\left(\begin{array}{c}m \dot V\\I \dot \omega + \omega I \omega \end{array}\right)=jachqT^t r\]
So $jachqt=MN$
