 \begin{table}[!ht]
  \begin{tabular}{|l|l|}
    \hline
    author  & V. Acary\\
    \hline
    date    & Sept, 20, 2011 \\ 
    \hline
    version &  \\
    \hline
  \end{tabular}
\end{table}



This section is devoted to the implementation and the study  of the algorithm. The interval of integration is $[0,T]$, $T>0$, and a grid $t_{0}=0$, $t_{k+1}=t_{k}+h$, $k \geq 0$, $t_{N}=T$ is constructed. The approximation of a function $f(\cdot)$ on $[0,T]$ is denoted as $f^{N}(\cdot)$, and is a piecewise constant function, constant on the intervals $[t_{k},t_{k+1})$. We denote $f^{N}(t_{k})$ as $f_{k}$. The time-step is $h>0$. 


\section{Various second  order dynamical systems with input/output relations}



\subsection{Lagrangian dynamical systems}


The class {\tt LagrangianDS}  defines  and computes a generic ndof-dimensional 
Lagrangian Non Linear Dynamical System of the form :

\begin{equation}
  \begin{cases}
    M(q,z) \dot v + N(v, q, z) + F_{Int}(v , q , t, z) = F_{Ext}(t, z) + p \\
    \dot q = v
  \end{cases}
\end{equation}
 where 
 \begin{itemize}
 \item  $q \in R^{ndof} $ is the set of the generalized
   coordinates, 
 \item $ \dot q =v \in R^{ndof} $ the velocity,
   i. e. the time derivative of the generalized coordinates
   (Lagrangian systems).
 \item $ \ddot q =\dot v \in R^{ndof} $ the
   acceleration, i. e. the second time derivative of the generalized
   coordinates.  
 \item $ p \in R^{ndof} $ the reaction forces due to
   the Non Smooth Interaction.  
 \item $ M(q) \in R^{ndof \times ndof}
   $ is the inertia term saved in the SiconosMatrix mass.  
 \item $
   N(\dot q, q) \in R^{ndof}$ is the non linear inertia term saved
   in the {\tt SiconosVector \_NNL}.  
 \item $ F_{Int}(\dot q , q , t) \in
   R^{ndof} $ are the internal forces saved in the SiconosVector
   fInt.  
 \item $ F_{Ext}(t) \in R^{ndof} $ are the external forces
   saved in the SiconosVector fExt.  
 \item $ z \in R^{zSize}$ is a
   vector of arbitrary algebraic variables, some sort of discrete
   state.
 \end{itemize}

 
  The equation of motion is also shortly denoted as:
  \begin{equation}
  M(q,z) \dot v = F(v, q, t, z) + p
\end{equation}
 
  where  $F(v, q, t, z) \in R^{ndof} $ collects the total forces
  acting on the system, that is 
  \begin{equation}
    F(v, q, t, z) =  F_{Ext}(t, z) -  NNL(v, q, z) + F_{Int}(v, q , t, z) 
\end{equation}

 This vector is stored in the  {\tt SiconosVector \_Forces  }  

\subsection{Fully nonlinear case}
Let us introduce the following system,
\begin{equation}
  \label{eq:FullyNonLinear}
  \begin{cases}
    M(q,z) \dot v = F(v, q, t, z) + p  \\
    \dot q = v \\
    y = h(t,q,\lambda) \\
    p = g(t,q,\lambda)
  \end{cases}
\end{equation}
where $\lambda(t) \in \RR^m$  and $y(t) \in \RR^m$ are  complementary variables related through a multi-valued mapping. According to the class of systems, we are studying, the function $F$ , $h$ and $g$ are defined by a fully nonlinear framework or by affine functions. This fully nonlinear case is not  implemented in Siconos yet. This fully general case is not yet implemented in Siconos.



\subsection{Lagrangian Rheonomous relations}

\begin{equation}
  \label{eq:RheonomousNonLinear}
  \begin{cases}
    M(q,z) \dot v = F(v, q, t, z) + p \\
    \dot q = v \\
    y = h(t,q) \\
    p = G(t,q)\lambda)
  \end{cases}
\end{equation}

\subsection{Lagrangian Scleronomous relations}

\begin{equation}
  \label{eq:ScleronomousNonLinear}
  \begin{cases}
    M(q,z) \dot v  = F(v, q, t, z) + p  \\
    \dot q = v \\
    y = h(q) \\
    p = G(q)\lambda
  \end{cases}
\end{equation}


\paragraph{Fully Linear case}

\begin{equation}
  \label{eq:FullyLinear}
  \begin{cases}
    M \dot v   +C v + Kq = F_{Ext}(t, z) + p  \\
    \dot q = v \\
    y = C q + e + D\lambda  + F z \\
    p = C^T\lambda
  \end{cases}
\end{equation}




\section{Moreau's Time--discretizations} 

\section{Schatzman--Paoli 'scheme and its linearizations}


\subsection{The scheme}
\begin{subnumcases}{}
  M(q_{k})(q_{k+1}-2q_{k}+q_{k-1})  - h^2 F(v_{k+\theta}, q_{k+\theta}, t_{k+theta})  =  p_{k+1},\quad\,\\ \notag\\ 
  v_{k+1}=\Frac{q_{k+1}-q_{k-1}}{2h}, \\ \notag \\
  y_{k+1} = h\left(\Frac{q_{k+1}+e q_{k-1}}{1+e}\right) \\
  p_{k+1}= G\left(\Frac{q_{k+1}+e q_{k-1}}{1+e}\right) \lambda_{k+1} \\
  0 \leq y_{k+1}  \perp\lambda_{k+1} \geq 0 .
\end{subnumcases}




\begin{ndrva}
Should we have 
  $$ v_{k+1}=\Frac{q_{k+1}-q_{k-1}}{2h}$$ or  $$ v_{k+1}=\Frac{q_{k+1}-q_{k}}{h}$$ ? This question is particularly important for the initialization and the proposed $\theta$-scheme
\end{ndrva}
\subsection{The Newton linearization}

Let us define the residu on $q$
\begin{equation}
  \label{eq:residu}
  \mathcal R(q) =   M(q)(q-2q_{k}+q_{k-1})  + h^2 F( (\theta v(q)+ (1-\theta) v_k),\theta q+ (1-\theta) q_k),  t_{k+\theta})  -  p_{k+1}
\end{equation}
with 
\begin{equation}
  \label{eq:residu-linq1}
  v(q) = \Frac{q-q_{k-1}}{2h}
\end{equation}
that is
\begin{equation}
  \label{eq:residu-linq2}
  \mathcal R(q) =   M(q)(q-2q_{k}+q_{k-1})  + h^2 F( (\theta \Frac{q-q_{k-1}}{2h} + (1-\theta) v_k),\theta q+ (1-\theta) q_k),  t_{k+\theta})   -  p_{k+1}
\end{equation}

Neglecting $\nabla_q  M(q)$ we get 
\begin{equation}
  \label{eq:iterationmatrix}
 \nabla_q \mathcal R(q^\nu) =   M(q^\nu) + h^2  \theta K(q^\nu,v^\nu) + \Frac 1 2 h  \theta C(q^\nu,v^\nu)
\end{equation}
and we  have to solve
\begin{equation}
  \label{eq:iterationloop}
 \nabla_q \mathcal R(q^\nu)(q^{\nu+1}-q^\nu) = -  \mathcal R(q^\nu) .
\end{equation}



\subsection{Linear version of the scheme}


\begin{subnumcases}{}
  M(q_{k+1}-2q_{k}+q_{k-1})  + h^2 (K q_{k+\theta}+ C v_{k+\theta})  =  p_{k+1},\quad\,\\ \notag\\ 
  v_{k+1}=\Frac{q_{k+1}-q_{k-1}}{2h}, \\ \notag \\
  y_{k+1} = h\left(\Frac{q_{k+1}+e q_{k-1}}{1+e}\right) \\
  p_{k+1}= G\left(\Frac{q_{k+1}+e q_{k-1}}{1+e}\right) \lambda_{k+1} \\
  0 \leq y_{k+1}  \perp\lambda_{k+1} \geq 0 .
\end{subnumcases}

Let us define the residu on $q$
\begin{equation}
  \label{eq:residu-linq}
  \mathcal R(q) =   M(q-2q_{k}+q_{k-1})  + h^2 (K(\theta q+ (1-\theta) q_k))+ C (\theta v(q)+ (1-\theta) v_k))  -  p_{k+1}
\end{equation}
with 
\begin{equation}
  \label{eq:residu-linq1}
  v(q) = \Frac{q-q_{k-1}}{2h}
\end{equation}
that is
\begin{equation}
  \label{eq:residu-linq2}
  \mathcal R(q) =   M(q-2q_{k}+q_{k-1})  + h^2 (K(\theta q+ (1-\theta) q_k)))+  h^2 C (\theta \Frac{q-q_{k-1}}{2h}+ (1-\theta) v_k))  -  p_{k+1}
\end{equation}

In this linear case, assuming that $q^0=q^\nu = q_k$, we get
\begin{equation}
  \label{eq:residu-linq2}
  \mathcal R(q^\nu) =   M(-q_{k}+q_{k-1})  + h^2 (K q_k)+  h^2 C (\theta \Frac{q_k-q_{k-1}}{2h}+ (1-\theta) v_k))  -  p_{k+1}
\end{equation}


\section{What about mixing {\tt OnestepIntegrator} in Simulation?}
\label{Sec:MisingOSI}
Let us consider that we have two simple linear Lagrangian Dynamical systems
\begin{equation}
  \label{eq:FullyLinear1}
  \begin{cases}
    M_1 \dot v_1  = F_{1,Ext}(t) + p_1   \\
    \dot q_1 = v_1 
  \end{cases}
\end{equation}
and
\begin{equation}
  \label{eq:FullyLinear1}
  \begin{cases}
    M_2 \dot v_2   = F_{2,Ext}(t) + p_2  \\
    \dot q_2 = v_2 \\
  \end{cases}
\end{equation}
These Dynamical systems (\ref{eq:FullyLinear1}) and (\ref{eq:FullyLinear1}) might numerically solved by choosing two different time--stepping schemes. Let us choose for instance Moreau's scheme for(\ref{eq:FullyLinear1}) 
\begin{equation}
  \label{eq:FullyLinear1-TS}
  \begin{cases}
    M_1 (v_{1,k+1}-v_{1,k})  = F_{1,Ext}(t_{k+1}) + p_{1,k+1}   \\
    q_{1,k+1} = q_{k}+ h  v_{1,k+\theta} 
  \end{cases}
\end{equation}
and Schatzman--Paoli's sheme for (\ref{eq:FullyLinear1}) 
\begin{equation}
  \label{eq:FullyLinear1-TS}
  \begin{cases}
    M_2(q_{2,k+1}-2q_{2,k}+q_{2,k-1})  = F_{2,Ext}(t_{k+1}) + p_{2,k+1}  \\
    v_{2,k+1} = \Frac{q_{2,k+1}-q_{2,k-1}}{2h} \\
  \end{cases}
\end{equation}


Let us consider known that we have a {\tt LagrangianLinearTIR} between this two DSs such that
\begin{equation}
  \label{eq:LTIR-2DS}
  \begin{array}{l}
  y = q_1-q_2 \geq 0 \\ \\
  p = \left[
  \begin{array}{c}
    1 \\
    -1
  \end{array}\right] \lambda
\end{array}
\end{equation}
and a complementarity condition
\begin{equation}
  \label{eq:CP}
  0\leq y \perp \lambda \geq 0
\end{equation}
Many questions are raised when we want to deal with the discrete systems:
\begin{itemize}
\item Which rules should we use for the discretization of~(\ref{eq:CP}) ?
  \begin{equation}
    \label{eq:CP-TS1}
    \text{ if } \bar y_{k+1}\leq 0, \text{ then }  0\leq \dot y _{k+1} + e \dot y_{k} \perp \hat \lambda_{k+1}\geq 0 
  \end{equation}
  or
  \begin{equation}
    \label{eq:CP-TS2}
    0\leq y _{k+1} + e y_{k-1} \perp \tilde \lambda_{k+1}\geq 0 
  \end{equation}
\item Should we assume that $y_{k+1} = q_{1,k+1}-q_{2,k+1}$ and $\dot y_{k+1} = v_{1,k+1}-v_{2,k+1}$
\item How can we link $\hat \lambda_{k+1}$ and  $\tilde \lambda_{k+1}$ with $p_{1,k+1}$ and $p_{2,k+1}$ ?
\end{itemize}

The third is the more difficult question and is seems that it is not reasonable to deal with two DS related by one interaction with different osi.In practice, this should be avoided in Siconos.




%%% Local Variables: 
%%% mode: latex
%%% TeX-master: "DevNotes"
%%% End: 
