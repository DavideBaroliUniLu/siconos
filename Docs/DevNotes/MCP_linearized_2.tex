


\subsection{Time--discretization of the the case~(\ref{eq:quatre}) } 

This case is implemented in Siconos with the relation FirstOrderType2R.

Starting from  (\ref{eq:quatre}), let us introduce a new notation, 
\begin{equation}
\begin{array}{l}
M \dot{x}(t) = f(x(t),t) + r(t)  \\[2mm]
y(t) = h(t,x(t),\lambda (t)) \\[2mm]
r(t) = g(t,\lambda (t) ) \\[2mm]
\end{array}
\label{sept-bis-bis}
\end{equation}
Let us now proceed with the time discretization of (\ref{sept-bis-bis}) by a fully implicit scheme : 
\begin{equation}
  \begin{array}{l}
    \label{eq:toto1}
     M x_{k+1} = M x_{k} +h\theta f(x_{k+1},t_{k+1})+h(1-\theta) f(x_k,t_k) + h \theta _r r(t_{k+1})
     + h(1-\theta _r)r(t_k)  \\[2mm]
     y_{k+1} =  h(t+1,x_{k+1},\lambda _{k+1}) \\[2mm]
     r_{k+1} = g(\lambda_{k+1},t+1)\\[2mm]
  \end{array}
\end{equation}


\subsection{Newton's linearization}


 \paragraph{Newton's linearization} The first line of the  problem~(\ref{eq:toto1}) can be written under the form of a residue $\mathcal R$ depending only on $x_{k+1}$ and $r_{k+1}$ such that 
\begin{equation}
  \label{eq:NL23}
  \mathcal R (x_{k+1},r _{k+1}) =0
\end{equation}
with $\mathcal R(x,r) = M (x - x_{k}) -h\theta f( x , t_{k+1}) - h(1-\theta)f(x_k,t_k) - h\theta _r r
- h(1-\theta_r)r_k$.
The solution of this system of nonlinear equations is sought as a limit of the sequence $\{ x^{\alpha}_{k+1},r^{\alpha}_{k+1} \}_{\alpha \in \NN}$ such that
 \begin{equation}
   \label{eq:NL27}
   \begin{cases}
     x^{0}_{k+1} = x_k \\ \\
     \mathcal R_L( x^{\alpha+1}_{k+1},r^{\alpha+1}_{k+1}) = \mathcal
     R(x^{\alpha}_{k+1},r^{\alpha}_{k+1})  + \left[ \nabla_{x} \mathcal
     R(x^{\alpha}_{k+1},r^{\alpha}_{k+1})\right] (x^{\alpha+1}_{k+1}-x^{\alpha}_{k+1} ) +
     \left[ \nabla_{x} \mathcal R(x^{\alpha}_{k+1},r^{\alpha}_{k+1})\right] (r^{\alpha+1}_{k+1} - r^{\alpha}_{k+1} ) =0
 \end{cases}
\end{equation}
The residu free is also defined (useful for implementation only):
\[\mathcal R _{free}(x) =  M (x - x_{k}) -h\theta f( x , t_{k+1}) - h(1-\theta)f(x_k,t_k)\]
\[\mathcal R (x,r) = \mathcal R _{free}(x)   - h\theta _r r - h(1-\theta_r)r_k\]
\[ \fbox{$\mathcal R^{\alpha}_{k+1} = \mathcal R (x^{\alpha}_{k+1},r^{\alpha}_{k+1}) = \mathcal R
_{free}(x^{\alpha}_{k+1},r^{\alpha}_{k+1} )  - h\theta _r r^{\alpha}_{k+1} - h(1-\theta_r)r_k$}\]
\[ \mathcal R _{free k+1} ^{\alpha} = \mathcal R
_{free}(x^{\alpha}_{k+1},r^{\alpha}_{k+1} )=\fbox{$M(x^{\alpha}_{k+1} - x_{k}) -h\theta f( x^{\alpha}_{k+1} , t_{k+1}) - h(1-\theta)f(x_k,t_k)$}\]
 
The computation of the Jacobian of $\mathcal R$ with respect to $x$, denoted by $M(x,\lambda)$ leads to 
\begin{equation}
   \label{eq:NL29}
   \begin{array}{l}
    W^{\alpha}_{k+1}= \nabla_{x} \mathcal R (x^{\alpha}_{k+1},r^{\alpha}_{k+1})= M - h  \theta \nabla_{x} f(  x^{\alpha}_{k+1}, t_{k+1} ).\\
 \end{array}
\end{equation}
At each time--step, we have to solve the following linearized problem,
\begin{equation}
   \label{eq:NL210}
    \mathcal R^{\alpha}_{k+1} + W^{\alpha}_{k+1} (x^{\alpha+1}_{k+1} -
    x^{\alpha}_{k+1}) - h \theta _r (r^{\alpha+1}_{k+1} - r^{\alpha}_{k+1} )  =0 ,
\end{equation}
that is

\begin{equation}
   \begin{array}{l}
 h \theta _r  r^{\alpha+1}_{k+1} = r_c + W^{\alpha}_{k+1} x^{\alpha+1}_{k+1}
 .\label{eq:NL211} 
 \end{array}
\end{equation}
with 
\begin{equation}
   \begin{array}{l}
r_c = h \theta _r r^{\alpha}_{k+1} - W^{\alpha}_{k+1} x^{\alpha}_{k+1} + \mathcal R
^{\alpha}_{k+1}=- W^{\alpha}_{k+1} x^{\alpha}_{k+1} + \mathcal R_{free k+1} ^{\alpha} - h(1-\theta_r)r_k
 \end{array}
   \end{equation}
so:
\[x^{\alpha+1}_{k+1} = h(W^{\alpha}_{k+1})^{-1}r^{\alpha+1}_{k+1} +(W^{\alpha}_{k+1})^{-1}(\mathcal R_{free k+1} ^{\alpha})+x^{\alpha}_{k+1}\]
\[\fbox{$x^{\alpha+1}_{k+1} = h(W^{\alpha}_{k+1})^{-1}g^{\alpha+1}_{k+1} +x_{free}$}\]
The matrix $W$ is clearly non singular for small $h$.
The same operation is performed with the second equation of (\ref{eq:toto1})
\begin{equation}
   \label{eq:NL29}
   \begin{array}{l}
      \mathcal R_y(x,y,\lambda)=y-h(t_{k+1},x,\lambda) =0\\ \\
      \mathcal R_{Ly}(x^{\alpha+1}_{k+1},y^{\alpha+1}_{k+1},\lambda^{\alpha+1}_{k+1}) = \mathcal
      R_{y}(x^{\alpha}_{k+1},y^{\alpha}_{k+1},\lambda^{\alpha}_{k+1}) +
      (y^{\alpha+1}_{k+1}-y^{\alpha}_{k+1})- \\ \qquad \qquad
      C^{\alpha}_{k+1}(x^{\alpha+1}_{k+1}-x^{\alpha}_{k+1}) - D^{\alpha}_{k+1}(\lambda^{\alpha+1}_{k+1}-\lambda^{\alpha}_{k+1})=0\\
      
 \end{array}
\end{equation}

$\mathcal R_y(x^{\alpha+1}_{k+1},y^{\alpha+1}_{k+1},\lambda^{\alpha+1}_{k+1})$ leading to the following linearized equation
\begin{equation}
  y^{\alpha+1}_{k+1} =  y^{\alpha}_{k+1}
  -\mathcal R^{\alpha}_{yk+1}+ \\
  C^{\alpha}_{k+1}(x^{\alpha+1}_{k+1}-x^{\alpha}_{k+1}) +
  D^{\alpha}_{k+1}(\lambda^{\alpha+1}_{k+1}-\lambda^{\alpha}_{k+1}). \label{eq:NL211y}
\end{equation}
with,
\begin{equation}
     \begin{array}{l}
  C^{\alpha}_{k+1} = \nabla_xh(t_{k+1}, x^{\alpha}_{k+1},\lambda^{\alpha}_{k+1} ) \\ \\
  D^{\alpha}_{k+1} = \nabla_{\lambda}h(t_{k+1}, x^{\alpha}_{k+1},\lambda^{\alpha}_{k+1})
 \end{array}
\end{equation}
and
\begin{equation}\fbox{$
\mathcal R^{\alpha}_{yk+1} = y^{\alpha}_{k+1} - h(x^{\alpha}_{k+1},\lambda^{\alpha}_{k+1},t_{k+1})$}
 \end{equation}

The same operation is performed with the thirdtheta equation of (\ref{eq:toto1})
\begin{equation}
   \label{eq:NL29}
   \begin{array}{l}
      \mathcal R_r(r,\lambda)=r-g(\lambda,t_{k+1}) =0\\ \\
      \mathcal R_{L\lambda}(r^{\alpha+1}_{k+1},\lambda^{\alpha+1}_{k+1}) = \mathcal
      R_{rk+1}^{\alpha} + (r^{\alpha+1}_{k+1} - r^{\alpha}_{k+1}) -
      B^{\alpha}_{k+1}(\lambda^{\alpha+1}_{k+1} -
      \lambda^{\alpha}_{k+1})=0\\ \\

      r^{\alpha+1}_{k+1} = - \mathcal R_{rk+1}^{\alpha} + r^{\alpha}_{k+1} + B^{\alpha}_{k+1}(\lambda^{\alpha+1}_{k+1} -
      \lambda^{\alpha}_{k+1})\\ \\
      \fbox{$r^{\alpha+1}_{k+1} = g(\lambda ^{\alpha}_{k+1},t_{k+1}) -B^{\alpha}_{k+1}
      \lambda^{\alpha}_{k+1} + B^{\alpha}_{k+1} \lambda^{\alpha+1}_{k+1}$}       
 \end{array}
\end{equation}
because the r residue is:
\begin{equation}
\mathcal
      R_{rk+1}^{\alpha} = r^{\alpha}_{k+1} - g(r^{\alpha}_{k+1},t_{k+1})
  \end{equation}
with,

\begin{equation}
     \begin{array}{l}
  B^{\alpha}_{k+1} = \nabla_{\lambda}g(r^{\alpha}_{k+1},t_{k+1})
 \end{array}
\end{equation}

Inserting (\ref{eq:NL211}), we get the following linear relation between $x^{\alpha+1}_{k+1}$ and
$\lambda^{\alpha+1}_{k+1}$, 

\begin{equation}
   \begin{array}{l}
    (W^{\alpha}_{k+1})x^{\alpha+1}_{k+1} = h \theta _r \left[ g(r^{\alpha}_{k+1},t_{k+1}) +
    B^{\alpha}_{k+1} (\lambda^{\alpha+1}_{k+1} - \lambda^{\alpha}_{k+1}) \right] -r_c \\ \\
    x^{\alpha+1}_{k+1} = h \theta _r (W^{\alpha}_{k+1})^{-1}\left[ g(r^{\alpha}_{k+1},t_{k+1}) + B^{\alpha}_{k+1} (\lambda^{\alpha+1}_{k+1} - \lambda^{\alpha}_{k+1}) \right] -(W^{\alpha}_{k+1})^{-1}(- W^{\alpha}_{k+1} x^{\alpha}_{k+1} + \mathcal R_{free k+1} ^{\alpha} - h(1-\theta_r)r_k) \\ \\
    x^{\alpha+1}_{k+1} =  x_p + h \theta _r (W^{\alpha}_{k+1})^{-1}  B^{\alpha}_{k+1}
    \lambda^{\alpha+1}_{k+1} 

   \end{array}
\end{equation}
with $\theta _r =1$:
\[(W^{\alpha}_{k+1})x^{\alpha+1}_{k+1}= hr^{\alpha+1}_{k+1}- \mathcal R_{free k+1} ^{\alpha}+W^{\alpha}_{k+1})x^{\alpha}_{k+1}\]
\[x^{\alpha+1}_{k+1}= h(W^{\alpha}_{k+1})^{-1}r^{\alpha+1}_{k+1}- (W^{\alpha}_{k+1})^{-1} \mathcal R_{free k+1} ^{\alpha}+x^{\alpha}_{k+1}\]
\[x^{\alpha+1}_{k+1}= h(W^{\alpha}_{k+1})^{-1}r^{\alpha+1}_{k+1}+x_{free}\]
with
\begin{equation}
x_p - x^{\alpha}_{k+1} =  (W^{\alpha}_{k+1})^{-1}  \left[ -r_c + h \theta _r( g(r^{\alpha}_{k+1},t_{k+1}) -
      B^{\alpha}_{k+1} \lambda^{\alpha}_{k+1} \right)]
  \end{equation}

\[ x_p - x^{\alpha}_{k+1} = \tilde x_{free} + h \theta _r(W^{\alpha}_{k+1})^{-1}( g(r^{\alpha}_{k+1},t_{k+1}) -
      B^{\alpha}_{k+1} \lambda^{\alpha}_{k+1} ) \]
      
\[    \tilde x_{free}= (W^{\alpha}_{k+1})^{-1}(-
      \mathcal R _{free k+1} ^{\alpha} +h(1-\theta _r)r_k)\]
      \[x_{free} = \tilde x_{free} + x^{\alpha}_{k+1}=\fbox{$-W^{-1}R_{free k+1} ^{\alpha} + x^{\alpha}_{k+1}$}\]
\[ \fbox{$x_p  = x_{free} + h \theta _r(W^{\alpha}_{k+1})^{-1}( g(\lambda ^{\alpha}_{k+1},t_{k+1}) -
      B^{\alpha}_{k+1} \lambda^{\alpha}_{k+1} )$} \]

    
Inserting (\ref{eq:NL211y}), we get the following linear relation between $y^{\alpha+1}_{k+1}$ and $\lambda^{\alpha+1}_{k+1}$, 
\begin{equation}
   \begin{array}{l}
 y^{\alpha+1}_{k+1} = y_p + \left[ h \theta _r C^{\alpha}_{k+1} (W^{\alpha}_{k+1})^{-1}  B^{\alpha}_{k+1} + D^{\alpha}_{k+1} \right]\lambda^{\alpha+1}_{k+1}
   \end{array}
\end{equation}

with 
\begin{equation}
\fbox{$y_p = y^{\alpha}_{k+1} -\mathcal R^{\alpha}_{yk+1} + C^{\alpha}_{k+1}(x_p -x^{\alpha}_{k+1}) -
D^{\alpha}_{k+1} \lambda^{\alpha}_{k+1} $}
\end{equation}
\begin{equation}
y_p = h(x^{\alpha}_{k+1},\lambda^{\alpha}_{k+1}) + C^{\alpha}_{k+1}(x_p -x^{\alpha}_{k+1}) -
D^{\alpha}_{k+1} \lambda^{\alpha}_{k+1} 
\end{equation}
\begin{equation}
y_p = h(x^{\alpha}_{k+1},\lambda^{\alpha}_{k+1}) - C^{\alpha}_{k+1}x^{\alpha}_{k+1}
+C^{\alpha}_{k+1} x_{free} +hC(W^{\alpha}_{k+1})^{-1}(g(\lambda
^{\alpha}_{k+1},t_{k+1})-B^{\alpha}_{k+1} \lambda^{\alpha}_{k+1}) -
D^{\alpha}_{k+1} \lambda^{\alpha}_{k+1} 
\end{equation}

\paragraph{Mixed linear complementarity problem (MLCP)}To summarize, the problem to be solved in each Newton iteration is:\\{
  \begin{minipage}[l]{1.0\linewidth}
    \begin{equation}
      \begin{cases}
      \begin{array}[l]{l}
        y^{\alpha+1}_{k+1} =   \tilde W^{\alpha}_{k+1}  \lambda^{\alpha+1}_{k+1} + b^{\alpha}_{k+1}
        \\ \\
        -y^{\alpha+1}_{k+1} \in N_{[l,u]}(\lambda^{\alpha+1}_{k+1} ). 
      \end{array}
      \label{eq:NL214}
      \end{cases}
    \end{equation}
  \end{minipage}
}
with $W\in \RR^{m\times m}$ and $b\in\RR^{m}$ defined by
\begin{equation}
  \label{eq:NL215}
 \begin{array}[l]{l}
   \tilde W^{\alpha}_{k+1} = h \theta _r C^{\alpha}_{k+1} (W^{\alpha}_{k+1})^{-1}  B^{\alpha}_{k+1} + D^{\alpha}_{k+1} \\
   b^{\alpha}_{k+1} = y_p
\end{array}
\end{equation}

The problem~(\ref{eq:NL214}) is equivalent to a Mixed Linear Complementarity Problem (MLCP) which can be solved under suitable assumptions by many linear complementarity solvers such as pivoting techniques, interior point techniques and splitting/projection strategies. The  reformulation into a standard MLCP follows the same line as for the MCP in the previous section. One obtains,
    \begin{equation}
      \begin{array}[l]{l}
        y^{\alpha+1}_{k+1} =   - W^{\alpha}_{k+1}  \lambda^{\alpha+1}_{k+1} + b^{\alpha}_{k+1}
        \\ \\
        (y^{\alpha+1}_{k+1})_i  = 0 \qquad \textrm{ for } i \in \{ 1..n\}\\[2mm]
        0 \leq  (\lambda^{\alpha+1}_{k+1})_i\perp (y^{\alpha+1}_{k+1})_i \geq 0 \qquad \textrm{ for } i \in \{ n..n+m\}\\
      \end{array}
      \label{eq:MLCP1} 
    \end{equation}

\paragraph{MLCP solvers.} As for MCP, there exists numerous methods to numerically solve MLCP. In the worst case when the matrix $W^{\alpha+1}_{k+1}$ has no special properties, the MCLP can be always solved by enumerative solvers for which various implementation can be found. With positivity, P-matrix  or co-positivity properties, standard methods for LCP\cite{Cottle.Pang.ea1992} can be straightforwardly extended. Among these methods, we can cite the family of projection/splitting methods, interior point methods and semi-smooth Newton methods (see \cite{acary2008} for an overview of various types of methods.).

\subsection{Implementation In siconos}

The initial residu x:
\[\mathcal   R^{0}_{k+1}= \mathcal   R^{0}_{free k+1}-hr^{0}_{k+1}\]
\[\mathcal   R^{0}_{k+1}= M(x^{0}_{k+1} - x_k) -h\theta f(x_k,t_{k+1})-h(1-\theta)f(x_k,t_k)-hr^{0}_{k+1}\]
\[\mathcal   R^{0}_{k+1}= -h\theta f(x_k,t_{k+1})-h(1-\theta)f(x_k,t_k)-hr^{0}_{k+1}\]
The initial residu y:
\[\mathcal   R^{0}_{y k+1} =y_k -h(x_k,\lambda _k,t_{k+1})\]
The initial residu r:
\[\mathcal   R^{0}_{r k+1} =r_k -g(\lambda _k,t_{k+1})\]

%\subsection{Algorithms} 

%We propose in this section two algorithms to sum-up the numerical implementation of the implicit
%Euler time--stepping scheme.
%The Algorithm~\ref{Algo:EulerSliding-MCP} describes the implementation
%with a generic MCP solvers and
%The Algorithm~\ref{Algo:EulerSliding-MCLP} describes the  numerical implementation of the algorithm with an external Newton linearization and a MCLP solver.


%\begin{algorithm}[htbp]
%   \begin{algorithmic}
% { \sf 
%    \REQUIRE $x_k, x^{\alpha+1}_{k+1},\lambda ^{\alpha+1}_{k+1} ,\lambda ^{\alpha}_{k+1},g^{\alpha}_{k+1} $ 
%    \ENSURE  $g^{\alpha+1}_{k+1}, r^{\alpha+1}_{k+1}, R^{\alpha+1}_{k+1},  Rfree^{\alpha+1}_{k+1}, Rr^{\alpha+1}_{k+1}$
%    \STATE $r^{\alpha+1}_{k+1} = g^{\alpha}_{k+1} +B^{\alpha}_{k+1} (\lambda^{\alpha+1}_{k+1} -\lambda ^{\alpha}_{k+1})$
%    \STATE $ Rfree^{\alpha+1}_{k+1} = M(x^{\alpha+1}_{k+1}-x_k)-h \theta f(x^{\alpha+1}_{k+1},t_{k+1})-h(1-\theta)f(x_k,t_k)$
%    \STATE $ R^{\alpha+1}_{k+1} = M(x^{\alpha+1}_{k+1}-x_k)-h \theta
%    f(x^{\alpha+1}_{k+1},t_{k+1})-h(1-\theta)f(x_k,t_k) -hr^{\alpha+1}_{k+1}$
%     \STATE $g^{\alpha+1}_{k+1} = g(\lambda ^{\alpha+1}_{k+1},t_{k+1})$
%     \STATE $Rr^{\alpha+1}_{k+1} =r^{\alpha+1}_{k+1} - g^{\alpha+1}_{k+1} $
%     
%     \STATE $ \sf$}
%   \end{algorithmic}
%   \caption{compute Residu x and r}  
% \label{Algo:ResiduXandR}
%\end{algorithm}

%\begin{algorithm}[htbp]
%   \begin{algorithmic}
% { \sf 
%    \REQUIRE $x^{\alpha+1}_{k+1}, \lambda^{\alpha+1}_{k+1}$ 
%    \ENSURE  $h^{\alpha+1}_{k+1},Ry^{\alpha+1}_{k+1}$
%     \STATE $ h^{\alpha+1}_{k+1}=h(x^{\alpha+1}_{k+1},\lambda^{\alpha+1}_{k+1},t_{k+1})$
%     \STATE $ Ry^{\alpha+1}_{k+1}=y^{\alpha+1}_{k+1} -h^{\alpha+1}_{k+1} $
%     \STATE $ \sf$}
%   \end{algorithmic}
%   \caption{compute Residu y }  
% \label{Algo:Residuy}
%\end{algorithm}
%
%
%\begin{algorithm}[htbp]
%   \begin{algorithmic}
% { \sf 
%    \REQUIRE System definition: $\sf f, g, h$ 
%    \REQUIRE $\sf x(0)$ the initial condition
%    \REQUIRE $\sf t_0, T$ time--integration interval
%    \REQUIRE $\sf h$ time--step
%    \REQUIRE $\sf \theta, \gamma$ numerical integration parameters 
%    \REQUIRE $\sf \varepsilon $ Newton's method tolerance
%    \ENSURE  $\sf (\{ x_k\}, \{ \lambda_k\},\{y_k \}), k \in\{1,2, \ldots \}$ 
%    \STATE $ \sf $
%    \STATE $\sf k \leftarrow 0;\quad  x^0_0 \leftarrow x(0);\quad \sf y^0_0 \leftarrow 0;\quad \sf
%    \lambda^0_0 \leftarrow 0; \sf g^0_0 \leftarrow g(0,0) $  
%    \STATE $ $
%    \STATE //\textit{ Time integration loop}
%     \WHILE {$\sf t_k < T$}
%         \STATE $x_k = x^{\alpha}_k$
%         \STATE $\sf \alpha \leftarrow 0;\quad\sf x^0_{k+1} ; \sf \lambda
%    ^0_{k+1}; \sf y^0_{k+1}; \sf g^{\alpha}_{k+1}$
%        \STATE  computeResiduXandR \\
%        \STATE  computeResiduY \\

%        \STATE //\textit{Newton's loop}
%          \WHILE {$\sf  \| R^{0}_{k+1}\|+\| Ry^{0}_{k+1}\|+\| Rr^{0}_{k+1}\| > \varepsilon$}  
%         \STATE  $x_{free}=x^{\alpha}_{k+1} -(W^{\alpha}_{k+1})^{-1}(\mathcal R^{alpha}_{free k+1})$
%         \STATE  $x_p=x_{free}+h(W^{\alpha}_{k+1})^{-1}(g^{\alpha}_{k+1} -B^{\alpha}_{k+1} \lambda ^{\alpha}_{k+1})$
%         \STATE $y_p=h^{\alpha}_{k+1} -
%          C^{\alpha}_{k+1}x^{\alpha}_{k+1}+C^{\alpha}_{k+1}x_{free}+hC^{\alpha}_{k+1}(W^{\alpha}_{k+1})^{-1}(g^{\alpha}_{k+1}-B\lambda
%          ^{\alpha}_{k+1})-D^{\alpha}_{k+1} \lambda ^{\alpha}_{k+1}$

%         \STATE Solve the MLCP, ie get $y^{\alpha+1}_{k+1}$ and $\lambda ^{\alpha}_{k+1}$.
%         \STATE $x^{\alpha+1}_{k+1} = x_p+h(W^{\alpha}_{k+1})^{-1}B^{\alpha}_{k+1} \lambda
%         ^{\alpha}_{k+1}$
%         \STATE call computeResiduXandR \\
%         \STATE call computeResiduY \\
%         \STATE $\alpha = \alpha +1$\\
%         \ENDWHILE
%      \STATE $x_k = x^{\alpha}_k$
%      \ENDWHILE
%     \STATE $ \sf$}
%   \end{algorithmic}
%   \caption{Implicit Euler time-discretization with an external Newton loop and a MLCP solver }  
% \label{Algo:EulerSliding-MCLP}
%\end{algorithm}

In the case of the system~(\ref{eq:deux}) with a affine function $f$ or $\theta =0$, the the MLCP matrix $W$ can be computed before the beginning of the time loop saving a lot of computing effort.  In the case of the system (\ref{eq:trois}) with $\theta=\gamma=0$, the MLCP matrix $W$ can be computed before the beginning of the Newton loop.
\clearpage


%%% Local Variables: 
%%% mode: latex
%%% TeX-master: "EulerSliding"
%%% End: 
