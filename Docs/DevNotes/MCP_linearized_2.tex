

\section{Newton's linearization of~(\ref{first-DS2})} 




Let us now proceed with the time discretization of (\ref{first-DS2}) by a fully implicit scheme : 
\begin{equation}
  \begin{array}{l}
    \label{eq:mlcp2-toto1-DS2}
     M x_{k+1} = M x_{k} +h\theta f(x_{k+1},t_{k+1})+h(1-\theta) f(x_k,t_k) + h \gamma r(t_{k+1})
     + h(1-\gamma)r(t_k)  \\[2mm]
     y_{k+1} =  h(t_{k+1},x_{k+1},\lambda _{k+1}) \\[2mm]
     r_{k+1} = g(\lambda_{k+1},t_{k+1})\\[2mm]
  \end{array}
\end{equation}


 \paragraph{Newton's linearization of the first line of~(\ref{eq:mlcp2-toto-DS2})} The first line of the  problem~(\ref{eq:mlcp2-toto-DS2}) can be written under the form of a residue $\mathcal R$ depending only on $x_{k+1}$ and $r_{k+1}$ such that 
\begin{equation}
  \label{eq:mlcp2-NL3}
  \mathcal R (x_{k+1},r _{k+1}) =0
\end{equation}
with $\mathcal R(x,r) = M(x - x_{k}) -h\theta f( x , t_{k+1}) - h(1-\theta)f(x_k,t_k) - h\gamma r
- h(1-\gamma)r_k$.
The solution of this system of nonlinear equations is sought as a limit of the sequence $\{ x^{\alpha}_{k+1},r^{\alpha}_{k+1} \}_{\alpha \in \NN}$ such that
 \begin{equation}
   \label{eq:mlcp2-NL7}
   \begin{cases}
     x^{0}_{k+1} = x_k \\ \\
     \mathcal R_L( x^{\alpha+1}_{k+1},r^{\alpha+1}_{k+1}) = \mathcal
     R(x^{\alpha}_{k+1},r^{\alpha}_{k+1})  + \left[ \nabla_{x} \mathcal
     R(x^{\alpha}_{k+1},r^{\alpha}_{k+1})\right] (x^{\alpha+1}_{k+1}-x^{\alpha}_{k+1} ) +
     \left[ \nabla_{r} \mathcal R(x^{\alpha}_{k+1},r^{\alpha}_{k+1})\right] (r^{\alpha+1}_{k+1} - r^{\alpha}_{k+1} ) =0
 \end{cases}
\end{equation}
\begin{ndrva}
  What about $r^0_{k+1}$ ?
\end{ndrva}

The residu free is also defined (useful for implementation only):
\[\mathcal R _{free}(x) \stackrel{\Delta}{=}  M(x - x_{k}) -h\theta f( x , t_{k+1}) - h(1-\theta)f(x_k,t_k),\]
which yields
\[\mathcal R (x,r) = \mathcal R _{free}(x)   - h\gamma r - h(1-\gamma)r_k.\]

\begin{equation}
  \mathcal R (x^{\alpha}_{k+1},r^{\alpha}_{k+1}) = \fbox{$\mathcal R^{\alpha}_{k+1} \stackrel{\Delta}{=}  \mathcal R
_{free}(x^{\alpha}_{k+1},r^{\alpha}_{k+1} )  - h\gamma r^{\alpha}_{k+1} - h(1-\gamma)r_k$}\label{eq:mlcp2-rfree-1}
\end{equation}

\[  \mathcal R
_{free}(x^{\alpha}_{k+1},r^{\alpha}_{k+1} )=\fbox{$ \mathcal R _{free k+1} ^{\alpha} \stackrel{\Delta}{=}  M(x^{\alpha}_{k+1} - x_{k}) -h\theta f( x^{\alpha}_{k+1} , t_{k+1}) - h(1-\theta)f(x_k,t_k)$}\]
 
The computation of the Jacobian of $\mathcal R$ with respect to $x$, denoted by $   W^{\alpha}_{k+1}$ leads to 
\begin{equation}
   \label{eq:mlcp2-NL9}
   \begin{array}{l}
    W^{\alpha}_{k+1} \stackrel{\Delta}{=} \nabla_{x} \mathcal R (x^{\alpha}_{k+1},r^{\alpha}_{k+1})= M - h  \theta \nabla_{x} f(  x^{\alpha}_{k+1}, t_{k+1} ).\\
 \end{array}
\end{equation}
At each time--step, we have to solve the following linearized problem,
\begin{equation}
   \label{eq:mlcp2-NL10}
    \mathcal R^{\alpha}_{k+1} + W^{\alpha}_{k+1} (x^{\alpha+1}_{k+1} -
    x^{\alpha}_{k+1}) - h \gamma (r^{\alpha+1}_{k+1} - r^{\alpha}_{k+1} )  =0 ,
\end{equation}
By using (\ref{eq:mlcp2-rfree-1}), we get
\begin{equation}
  \label{eq:mlcp2-rfree-2}
  \mathcal R
_{free}(x^{\alpha}_{k+1},r^{\alpha}_{k+1} )  - h\gamma r^{\alpha+1}_{k+1} - h(1-\gamma)r_k  + W^{\alpha}_{k+1} (x^{\alpha+1}_{k+1} -
    x^{\alpha}_{k+1})  =0 
\end{equation}

%\fbox
{
  \begin{equation}
    \boxed{ x^{\alpha+1}_{k+1} = h(W^{\alpha}_{k+1})^{-1}r^{\alpha+1}_{k+1} +x^\alpha_{free}}
  \end{equation}
}
with :
\begin{equation}
  \boxed{x^\alpha_{free}\stackrel{\Delta}{=}x^{\alpha}_{k+1}-(W^{\alpha}_{k+1})^{-1}\mathcal (R_{freek+1}^{\alpha} \textcolor{red}{- h(1-\gamma) r_k})\label{eq:mlcp2-rfree-12}}
\end{equation}

The matrix $W$ is clearly non singular for small $h$.




% that is

% \begin{equation}
%    \begin{array}{l}
%  h \gamma  r^{\alpha+1}_{k+1} = r_c + W^{\alpha}_{k+1} x^{\alpha+1}_{k+1}
%  .\label{eq:mlcp2-NL11} 
%  \end{array}
% \end{equation}
% with 
% \begin{equation}
%    \begin{array}{l}
% r_c \stackrel{\Delta}{=} h \gamma r^{\alpha}_{k+1} - W^{\alpha}_{k+1} x^{\alpha}_{k+1} + \mathcal R
% ^{\alpha}_{k+1}=- W^{\alpha}_{k+1} x^{\alpha}_{k+1} + \mathcal R_{free k+1} ^{\alpha} - h(1-\gamma)r_k\\ \\
% \end{array}
% \end{equation}
% \begin{equation}
%    \begin{array}{l}
% \mathcal R ^{\alpha}_{k+1}=M( x^{\alpha}_{k+1} - x_k) -h \theta f(x^{\alpha}_{k+1})-h(1-\theta)f(x_k)
% - h \gamma r^{\alpha}_{k+1} -h(1- \gamma)r_k
%  \end{array}
%    \end{equation}
% \[x^{\alpha+1}_{k+1} = h(W^{\alpha}_{k+1})^{-1}r^{\alpha+1}_{k+1} +(W^{\alpha}_{k+1})^{-1}(\mathcal
% R_{free k+1} ^{\alpha})+x^{\alpha}_{k+1}\]


 \paragraph{Newton's linearization of the second  line of~(\ref{eq:mlcp2-toto1-DS2})}
The same operation is performed with the second equation of (\ref{eq:mlcp2-toto1-DS2})
\begin{equation}
  \begin{array}{l}
    \mathcal R_y(x,y,\lambda)=y-h(t_{k+1},x,\lambda) =0\\ \\
  \end{array}
\end{equation}
which is linearized as
\begin{equation}
  \label{eq:mlcp2-NL9}
  \begin{array}{l}
    \mathcal R_{Ly}(x^{\alpha+1}_{k+1},y^{\alpha+1}_{k+1},\lambda^{\alpha+1}_{k+1}) = \mathcal
    R_{y}(x^{\alpha}_{k+1},y^{\alpha}_{k+1},\lambda^{\alpha}_{k+1}) +
    (y^{\alpha+1}_{k+1}-y^{\alpha}_{k+1})- \\[2mm] \qquad  \qquad \qquad \qquad  \qquad \qquad
    C^{\alpha}_{k+1}(x^{\alpha+1}_{k+1}-x^{\alpha}_{k+1}) - D^{\alpha}_{k+1}(\lambda^{\alpha+1}_{k+1}-\lambda^{\alpha}_{k+1})=0
  \end{array}
\end{equation}

This leads to the following linear equation
\begin{equation}
  \boxed{y^{\alpha+1}_{k+1} =  y^{\alpha}_{k+1}
  -\mathcal R^{\alpha}_{yk+1}+ \\
  C^{\alpha}_{k+1}(x^{\alpha+1}_{k+1}-x^{\alpha}_{k+1}) +
  D^{\alpha}_{k+1}(\lambda^{\alpha+1}_{k+1}-\lambda^{\alpha}_{k+1})}. \label{eq:mlcp2-NL11y}
\end{equation}
with,
\begin{equation}
     \begin{array}{l}
  C^{\alpha}_{k+1} = \nabla_xh(t_{k+1}, x^{\alpha}_{k+1},\lambda^{\alpha}_{k+1} ) \\ \\
  D^{\alpha}_{k+1} = \nabla_{\lambda}h(t_{k+1}, x^{\alpha}_{k+1},\lambda^{\alpha}_{k+1})
 \end{array}
\end{equation}
and
\begin{equation}\fbox{$
\mathcal R^{\alpha}_{yk+1} \stackrel{\Delta}{=} y^{\alpha}_{k+1} - h(x^{\alpha}_{k+1},\lambda^{\alpha}_{k+1})$}
 \end{equation}
 \paragraph{Newton's linearization of the third  line of~(\ref{eq:mlcp2-toto1-DS2})}
The same operation is performed with the third equation of (\ref{eq:mlcp2-toto1-DS2})
\begin{equation}
  \begin{array}{l}
    \mathcal R_r(r,x,\lambda)=r-g(\lambda,t_{k+1}) =0\\ \\  \end{array}
\end{equation}
which is linearized as
\begin{equation}
  \label{eq:mlcp2-NL9}
  \begin{array}{l}
      \mathcal R_{L\lambda}(r^{\alpha+1}_{k+1},x^{\alpha+1}_{k+1},\lambda^{\alpha+1}_{k+1}) = \mathcal
      R_{rk+1}^{\alpha} + (r^{\alpha+1}_{k+1} - r^{\alpha}_{k+1}) - B^{\alpha}_{k+1}(\lambda^{\alpha+1}_{k+1} -
      \lambda^{\alpha}_{k+1})=0
    \end{array}
  \end{equation}
\begin{equation}
  \label{eq:mlcp2-rrL}
  \begin{array}{l}
    \boxed{r^{\alpha+1}_{k+1} = g(x ^{\alpha}_{k+1},\lambda ^{\alpha}_{k+1},t_{k+1}) -B^{\alpha}_{k+1}
      \lambda^{\alpha}_{k+1} + B^{\alpha}_{k+1} \lambda^{\alpha+1}}       
  \end{array}
\end{equation}
with,
\begin{equation}
     \begin{array}{l}
  B^{\alpha}_{k+1} = \nabla_{\lambda}g(x^{\alpha}_{k+1},\lambda ^{\alpha}_{k+1},t_{k+1})
 \end{array}
\end{equation}
and the  residue for $r$:
\begin{equation}
\boxed{\mathcal
      R_{rk+1}^{\alpha} = r^{\alpha}_{k+1} - g(\lambda ^{\alpha}_{k+1},t_{k+1})}
  \end{equation}


\paragraph{Reduction to a linear relation between  $x^{\alpha+1}_{k+1}$ and
$\lambda^{\alpha+1}_{k+1}$}

Inserting (\ref{eq:mlcp2-rrL}) into~(\ref{eq:mlcp2-rfree-12}), we get the following linear relation between $x^{\alpha+1}_{k+1}$ and
$\lambda^{\alpha+1}_{k+1}$, 

\begin{equation}
   \begin{array}{l}
     x^{\alpha+1}_{k+1} = h\gamma(W^{\alpha}_{k+1} )^{-1}\left[g(x^{\alpha}_{k+1},\lambda^{\alpha}_{k+1},t_{k+1}) +
    B^{\alpha}_{k+1} (\lambda^{\alpha+1}_{k+1} - \lambda^{\alpha}_{k+1}) \right ] +x^\alpha_{free}
\end{array}
\end{equation}
that is 
\begin{equation}
  \begin{array}{l}
   x^{\alpha+1}_{k+1} =x_p + h \gamma (W^{\alpha}_{k+1})^{-1}    B^{\alpha}_{k+1} \lambda^{\alpha+1}_{k+1}
   \end{array}
\end{equation}
with 
\begin{equation}
  \boxed{x_p \stackrel{\Delta}{=}  h\gamma(W^{\alpha}_{k+1} )^{-1}\left[g(x^{\alpha}_{k+1},\lambda^{\alpha}_{k+1},t_{k+1}) +
    -B^{\alpha}_{k+1} (\lambda^{\alpha}_{k+1}) \right ] +x^\alpha_{free}}
\end{equation}


We get the linear relation
\begin{equation}
  \label{eq:mlcp2-rfree-13}
  \begin{array}{l}
 \boxed{   x^{\alpha+1}_{k+1}\stackrel{\Delta}{=} x_p + \left[ h \gamma (W^{\alpha}_{k+1})^{-1}    B^{\alpha}_{k+1} \lambda^{\alpha+1}_{k+1}\right]}
   \end{array}
\end{equation}




\paragraph{Reduction to a linear relation between  $y^{\alpha+1}_{k+1}$ and
$\lambda^{\alpha+1}_{k+1}$}

Inserting (\ref{eq:mlcp2-rfree-13}) into (\ref{eq:mlcp2-NL11y}), we get the following linear relation between $y^{\alpha+1}_{k+1}$ and $\lambda^{\alpha+1}_{k+1}$, 
\begin{equation}
   \begin{array}{l}
 y^{\alpha+1}_{k+1} = y_p + \left[ h  C^{\alpha}_{k+1} ( W^{\alpha}_{k+1})^{-1}  B^{\alpha}_{k+1} + D^{\alpha}_{k+1} \right]\lambda^{\alpha+1}_{k+1}
   \end{array}
\end{equation}
with 
\begin{equation}\boxed{
y_p = y^{\alpha}_{k+1} -\mathcal R^{\alpha}_{yk+1} + C^{\alpha}_{k+1}(x_q) -
D^{\alpha}_{k+1} \lambda^{\alpha}_{k+1} }
\end{equation}
\textcolor{red}{
  \begin{equation}
    \boxed{ x_q= x_p - x^{\alpha}_{k+1}\label{eq:mlcp2-xqq}}
  \end{equation}
}





\clearpage


%%% Local Variables: 
%%% mode: latex
%%% TeX-master: "DevNotes"
%%% End: 
