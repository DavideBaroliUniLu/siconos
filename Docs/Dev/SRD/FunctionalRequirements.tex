\subsection{\ac{siconos}/Numerics}
\begin{longtable}{%
    |>{\columncolor[gray]{.8}}p{0.1\textwidth}%
    |>{\columncolor[gray]{.95}}p{0.9\textwidth}|}
  \hline
  \rowcolor[gray]{.8}   SR. Id. & Software requirements description \\
  \hline 
  \hline
  & \textbf{General  requirements (\ac{siconos}/Numerics)}\\
  \hline 
  F.1.001 & The module \ac{siconos}/Numerics must provide basic algebra objects (vector, matrices, quaternions) relying on a matrix template library \\
  F.1.002 & The module \ac{siconos}/Numerics must provide  high performance methods for  basic vector and matrix operations\\
  F.1.003 & The module \ac{siconos}/Numerics must provide various storage methods for matrices:
  \begin{enumerate}
  \item Dense(full)
  \item Band
  \item Skyline
  \item Sparse
  \end{enumerate}\\
 \hline 
 & \textbf{ Numerical functionalities  requirements (\ac{siconos}/Numerics)}\\
  \hline
  F.1.010 & The module  \ac{siconos}/Numerics has to perform basic linear algebra computations  relying on standard libraries for numerical computation (e.g. LAPACK):
  \begin{enumerate}
  \item Solution of linear system 
  \item Eigenvalue and eigenvectors problem
  \item Singular Value Decomposition
  \end{enumerate}
  \\
  F.1.011 & The module \ac{siconos}/Numerics has to perform solution for mathematical programming problems:
  \begin{enumerate}
  \item Linear Complementarity Problem (LCP) (direct and iterative solutions) 
  \item Quadratic problem (QP) 
  \item Non Linear  Complementarity Problem (NCP) 
  \item General non smooth optimization problem
  \end{enumerate}\\
  F.1.012 & The module \ac{siconos}/Numerics has to perform basic computation for smooth time integration (one-step integration)  \\
  F.1.013 & The module \ac{siconos}/Numerics has to perform root finding for non linear smooth equations (Newton's method)\\
  F.1.014 & The module \ac{siconos}/Numerics has to perform root finding for non smooth (generalized) equations  \\
  F.1.015 & The module \ac{siconos}/Numerics has to perform numerical, analytical and automatic differentiation\\
  \hline

  & \textbf{Interface  requirements (\ac{siconos}/Numerics)}\\
  \hline 
   F.1.020 & The module \ac{siconos}/Numerics must provide a common interface to various methods dedicated to one specific type of numerical problem\\
  \hline
  \caption{\ac{siconos}/Engine. Software Requirements}\\
\end{longtable}



%-------------------------------------------------------------------------------------------------------------------------------%
\subsection{\ac{siconos}/Engine}
\begin{longtable}{%
    |>{\columncolor[gray]{.8}}p{0.1\textwidth}%
    |>{\columncolor[gray]{.95}}p{0.9\textwidth}|}
  \hline
  \rowcolor[gray]{.8}   SR. Id. & Software requirements description \\
  \hline 
  \hline
  & \textbf{ General requirements (\ac{siconos}/Engine)}\\
  \hline
  F.2.000 & The s/w shall propose a set of  canonical models to formalize physical process  (Model Formalization) \\
  F.2.001 & The s/w shall propose a set of numerical strategies to simulate the  canonical models (Numerical model)\\
  F.2.002 & The s/w  must be based on \ac{siconos}/Numerics  for the basic  scientific computation \\
  \hline
  & \textbf{Model Formalization requirements }\\
  \hline
  F.2.010  &  The s/w shall  define and describe several canonical model for general NSDS \\
  F.2.011  &  The s/w shall  define and describe several canonical model for SDS\\
  F.2.012  &  The s/w shall  define and describe several canonical model for relations\\
  F.2.013  &  The s/w shall  define and describe several canonical model for Non Smooth Laws\\  
  F.2.014  &  The s/w shall  define and describe a canonical model for LCS  \\
  F.2.015  &  The s/w shall  define and describe a canonical model for Lagrangian dynamical system with constraints \\
  F.2.016  &  The s/w shall  define and describe a canonical model for Piecewise Smooth systems \\ 
  F.2.017  &  The s/w shall  define and describe a canonical model for Higher order sweeping process\\ 
  F.2.018  &  The s/w shall  define and describe a canonical model for Projected dynamical systems \\ 
  F.2.019  &  The s/w shall  define and describe a canonical model for Unilateral differential inclusions \\
  F.2.020  &  The s/w shall  define and describe a canonical model for Differential variational inequalities  \\
  F.2.021  &  The s/w shall  define and describe a canonical model for Discrete time systems\\
  F.2.022  &  The s/w shall  define and describe a canonical model for Linear Time invariant System (LTI)\\ 
  F.2.023  &  The s/w shall  define and describe a canonical model for Non Linear System \\
  F.2.024  &  The s/w shall  define and describe a canonical model for Lagrangian system\\
  F.2.025  &  The s/w shall  define and describe a canonical model for Implicit System and differential Algebraic system\\
  F.2.026  &  The s/w shall  define and describe a canonical model for Linear Time invariant relation  \\  
  F.2.027  &  The s/w shall  define and describe a canonical model for Lagrangian relations (Jacobians)\\  
  F.2.028  &  The s/w shall  define and describe a canonical model for Non linear relations            \\  
  F.2.029  &  The s/w shall  define and describe a canonical model for Complementarity problem   \\  
  F.2.030  &  The s/w shall  define and describe a canonical model for Relay system              \\  
  F.2.031  &  The s/w shall  define and describe a canonical model for Friction-type law\\ 
  & \\
  F.2.040  &  The Model Formalization part must provide all of the ingredients for the numerical strategies part \\
  F.2.041  &  The model Formalization shall provide several  translation tools between the  canonical models if possible. (general NSDS, SDS, and non smooth laws) \\
  F.2.042  &   The s/w shall propose mechanisms to define the canonical model through C F77 functions. This is the Basic plug-in \\
  F.2.043  &   The s/w shall propose mechanisms (Plug-ins) to define the canonical model through 
  \begin{enumerate}
  \item \ac{matlab} functions 
  \item \ac{scilab} functions
  \end{enumerate}\\
  F.2.044  &   The s/w shall propose mechanisms (Plug-ins) to define the canonical model through Existing modelling software :
  \begin{enumerate}
  \item \ac{lmgc90} Plug-in
  \item \ac{modelica} Plug-in
  \item \ac{scicos} Plug-in
  \item \ac{simulink} Plug-in
  \end{enumerate} \\
  F.2.045  &   The s/w shall propose several tools for the analysis of the coherence of the  input\\
 \hline
  & \textbf{Numerical strategies requirements }\\
  \hline
  F.2.100  & The s/w shall propose a Time--stepping scheme\\
  F.2.101  & The s/w shall propose a Event--Driven scheme\\
  F.2.102  & The s/w shall propose a formalization of a one-step basic problem\\
  F.2.103  & The s/w shall propose an evaluation or prediction of the relations at discrete time\\
  F.2.104  & The s/w shall propose an input method for specific parameters for numerical strategies\\
  F.2.105  & The s/w shall propose an output method for specific parameters for numerical strategies\\
  \hline
  & \textbf{Interface with \ac{siconos}/Numerics }\\
  \hline
  F.2.200  & The module \ac{siconos}/Engine must use the interface proposed by the  \ac{siconos}/Numerics module for the numerical computations \\
  \hline
  & \textbf{Data managing}\\
  \hline
  F.2.300  &  The software must be able to call functions of exiting scientific softwares to read their own data files. Only \ac{lmgc90} files will be called in the first version of the plate-forme. \\
  F.2.301  & The software must allow users to save a NSDS in files under a formalized form. \\
  F.2.302  & The software must be able to proceed to several files formats conversions.
                \begin{enumerate}
                \item \ac{lmgc90} file to XML file.
                \item XML file to external software file (GNUplot).
                \end{enumerate}\\ 
  F.2.303  & The software must allow users to trace states of a modelised system during a simulation. \\
  F.2.304  & The software must provide a system of backup and restart of a simulation, using XML files. \\
  \caption{\ac{siconos}/Engine. Software Requirements}\\
\end{longtable}



%-------------------------------------------------------------------------------------------------------------------------------%
\subsection{\ac{siconos}/Front-End}
\begin{longtable}{%
|>{\columncolor[gray]{.8}}p{0.1\textwidth}%
|>{\columncolor[gray]{.95}}p{0.9\textwidth}|}
\hline
\rowcolor[gray]{.8}   SR. Id. & Software requirements description \\
\hline 
   & \textbf{ General requirements (SICONOS/Front-End)}\\
   \hline
   F.3.000 & The \ac{siconos}/Front-End shall propose a C++  API  of the major functionalities of the \ac{siconos}/Engine (Functionalities F.2.xxx)  . In particular, this API will be wrapped to a object--oriented scripting language). \\
   F.3.001 & The \ac{siconos}/Front-End shall propose a \ac{scilab}  API  of the major functionalities of the \ac{siconos}/Engine (Functionalities F.2.xxx)  . In particular, this API will be  called trough the \ac{scilab} software. \\ 
   F.3.002 & The \ac{siconos}/Front-End shall propose a \ac{matlab}  API  of the major functionalities of the \ac{siconos}/Engine (Functionalities F.2.xxx)  . In particular, this API will be  called trough the \ac{matlab} software. \\
   F.3.003 & The \ac{siconos}/Front-End shall propose a C++  API well-suited (completeness and granularity) for the the analysis (\ac{siconos}/Analysis), the control (\ac{siconos}/Control).    \\
\hline
\caption{\ac{siconos}/Front-End. Software Requirements}\\
\end{longtable}
\subsection{SICONOS/Control.}

The specific requirements for the \ac{siconos}/Control will be defined later. 
\begin{longtable}{%
|>{\columncolor[gray]{.8}}p{0.1\textwidth}%
|>{\columncolor[gray]{.95}}p{0.9\textwidth}|}
\hline
\rowcolor[gray]{.8}   SR. Id. & Software requirements description \\
\hline 
   & \textbf{ General requirements (\ac{siconos}/Control.)}\\
   \hline
   F.4.000 & \\
\hline
\caption{\ac{siconos}/Control. Software Requirements}\\
\end{longtable}
\subsection{\ac{siconos}/Analysis.}

The specific requirements for the \ac{siconos}/Analysis will be defined later.
\begin{longtable}{%
|>{\columncolor[gray]{.8}}p{0.1\textwidth}%
|>{\columncolor[gray]{.95}}p{0.9\textwidth}|}
\hline
\rowcolor[gray]{.8}   SR. Id. & Software requirements description \\
\hline 
   & \textbf{ General requirements (\ac{siconos}/Analysis.)}\\
   \hline
   F.5.000 & \\
\hline
\caption{\ac{siconos}/Analysis. Software Requirements}\\
\end{longtable}
\subsection{\ac{siconos}/IMSE.}
The specific requirements for the \ac{siconos}/IMSE will be defined later.

\begin{longtable}{%
|>{\columncolor[gray]{.8}}p{0.1\textwidth}%
|>{\columncolor[gray]{.95}}p{0.9\textwidth}|}
\hline
\rowcolor[gray]{.8}   SR. Id. & Software requirements description \\
\hline 
   & \textbf{ General requirements (\ac{siconos}/IMSE.)}\\
   \hline
   F.6.000 & \\
\hline
\caption{\ac{siconos}/IMSE. Software Requirements}\\
\end{longtable}



 
%%% Local Variables: 
%%% mode: latex
%%% TeX-master: "../report"
%%% End: 
