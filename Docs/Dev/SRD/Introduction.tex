\section{Purpose and scope of this document}
\label{Sec:SRD-Purpose-Scope}
The \ac{srd} is a contractual document which aims to define precisely the software to realize. It describes functionalities and characteristics of the product and constraints of development and exploitation. It is addressed to users and software framework builders. \\
\\
This document will be used as basis : 
\begin{itemize}
\item for the evaluation of the final product,
\item for the editing of test plan,
\item for the editing of the following documents : 
        \begin{itemize}
        \item \ac{add}.
        \item \ac{ddd}.
        \end{itemize}
\end{itemize}


\section{Context of the \ac{siconos} platform}
\label{Sec:SRD-Context}


\subsection{Scope}
\label{Sec:SRD-Scope}


This \ac{srd} concerns the \ac{kernel} and the \ac{imse}.




\section{References}
\label{Sec:SRD-References}
\begin{itemize}
\item \textbf{"Guide to the software requirements definition phase", PSS-05-03, ESA 1995}
\item "Guide to software requirements specifications", standard 830 - 1984,IEEE.
\item "Cours de Conduite de Projets Logiciels", Ioannis Parissis, Master 2 Pro GI 2003-2004.
\item Draft version of SRD of SICONOS / WP2, January 2004.
\item Templates and Guides, Roger Pissard-Gibollet, INRIA

\item \ac{SICONOS} Contract Number IST-2001-37172 and annexes
\item \ac{SICONOS} web site : http://maply.univ-lyon1.fr/siconos/
\item \ac{matlab} web site : http://www.mathworks.com/
\item \ac{scilab} web site : http://www-rocq.inria.fr/scilab/ 

\end{itemize}




\section{Overview}
%\begin{ndrva}
%  \begin{itemize}
%  \item Expliquer en deux mots l'objet de ce  travail de sp�cification
%  \item Donner le plan du SRD
%  \end{itemize}
%\end{ndrva}
This specification of requirements is composed of two phases :
\begin{enumerate}
\item A general description of the project. This part explains the purpose of this project and its integration into European project \ac{SICONOS}. is can be found in the \ac{gpd} document.
\item The problem definition phase. The scope of the software must be defined. Specific user requirements must be identified and documented. The section has to provide a general description of what the user expects the software to do. It should state the specific user requirements as clearly and consistently as possible. 
\end{enumerate}


