\section{Purpose and scope of this document}
\label{Sec:GPD-Purpose-Scope}
The \ac{gpd} aims at presenting a general description of the Siconos platform project. It describes, in a very general way, the major objectives of the platform and the major constraints of development and exploitation. It is addressed to users and software framework builders. \\
\\


\section{Context of the \ac{siconos} platform}
\label{Sec:GPD-Context}


\subsection{Scope}
\label{Sec:SRD-Scope}

This project belongs to the European project \ac{SICONOS}, dedicated to the modeling, the control the analysis and the simulation of the so-called ``Non Smooth Dynamical system''\acs{nsds}\footnote{ More details on this European project can be found on the website of the Project : \texttt{http://siconos.inrialpes.fr}}.

The design and the development of the \ac{siconos} platform which is one of the major objective of this European project, is told the work package 2 (``Numerical methods and algorithms'')  in confidence\footnote{ More details on this European project can be found on the website of the WP2 : \texttt{http://siconos.inrialpes.fr/software}}. 

\subsection{Main objectives and motivations}
The first version of the objectives can be found in the Annex 1 of the proposition of the European Project.

  Besides the standard features which are required for a software of scientific computing, the objectives of this project are of the following :
\begin{itemize}
\item To provide a common framework (modeling and simulation tools) for non smooth problems present in various scientific fields : applied Mathematics, Mechanics, Robotics, Electrical networks, etc. 
\item To be able to rely on existing developments, as well for the modeling tools as for the simulation tools. The platform will not re-implement the dedicated tools, which are already used  for the modelling of specific systems in various fields,but will provide a framework to their integration. In the same way, we want to re-use improved numerical low-level routines.
\item To support exchanges and comparisons of methods between researchers.
\item To disseminate the know-how to other fields of research and industry.
\item To take into account the diversity of users (end users,  experts, numericians, software developers, framework builders, industrial users, etc ... ).
\item To set up standards in terms of modelling of the \ac{nsds}.
\item To ensure software quality by the use of modern software design methods.
\end{itemize}



\subsection{Exploitation context}
The \ac{siconos} platform should be used as well by research academic teams as by the industries, interested in \ac{nsds}. As we said earlier, the platform must rely on existing habits of the community if users, that we want to reach. The exploitation must take into account  these hard constraints.

 To fulfil the constraints on the existing habits of a variety of users, the user interface must be implemented in three ways : 
\begin{itemize}
\item An expert and proper interface must be developed, allowing the access of the maximum of the functionalities of the platform. A accurate definition of an API of the platform based on all public method of the contained classes is unavoidable. A interface to this API in a scripting(interpreted) language should be suitable. 
\item An \ac{xxxlab} interface, for instance for  \ac{scilab}\footnote{A free scientific software package for numerical computations developed at \ac{inria} too and clone to \ac{matlab}, \texttt{www.scilab.org}} and for \ac{matlab}.
\item Finally, a graphical  user interface (GUI) embedded in a integrated modeling and simulation environment.
\end{itemize}

To fulfil the constraints to re-use the existing development and software, the product must implement this two functionalities :
\begin{itemize}
\item To have a user plug-in interface to use in an easy way the code which already model, non smooth dynamical. This users plug-in system intend to provide to the user the possibility to input his data without recompiling the platform
\item To have a expert plug-in system, which allows us to specify the behaviour of a generic type of \ac{nsds} in the platform by loading dynamically new types of derived systems.
\end{itemize}


\section{References}
\label{Sec:GPD-References}
\begin{itemize}
\item "Guide to software requirements specifications", standard 830 - 1984,IEEE
\item "Templates and Guides", Roger Pissard-Gibollet, INRIA
\item \ac{SICONOS} Contract Number IST-2001-37172 and annexes
\item \ac{SICONOS} web site : \texttt{http://siconos.inrialpes.fr (/software)}
\end{itemize}




