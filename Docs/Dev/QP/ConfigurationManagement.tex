\section{Tools}

This section gives an overview of the main tools used for the project developement.

\subsection{CVS}

As an introduction, this is a short presentation of CVS, quotated from CVS official web site  (https://www.cvshome.org/new\_users.html) :\\

\textit{CVS is the Concurrent Versions System, the dominant open-source network-transparent version control system. CVS is useful for everyone from individual developers 
  to large, distributed teams:
  \begin{itemize}
  \item Its client-server access method lets developers access the latest code from anywhere there's an Internet connection.
  \item Its unreserved check-out model to version control avoids artificial conflicts common with the exclusive check-out model.
  \item Its client tools are available on most platforms.
  \end{itemize}
}

\textit{CVS is used by popular open-source projects like Mozilla, the GIMP, XEmacs, KDE, and GNOME.}\\

This section is just a reminder of the particular rules of using CVS in \ac{siconos}. If you want to learn more about CVS, please see the following web-sites :
http://www.cvshome.org and http://www.gnu.org.\\

The following rules have to be respected by all users of \ac{siconos} CVS server :
\begin{itemize}
\item Write changing logs in english
\item Except very particular cases, commit only sources which succesfully compiles (and pass tests for the code). This rule is valid as much for documents as source code. 
\item Correct possible conflicts before committing your work.
\item Official documents of WP2 are stored on CVS server, under a branch named by the code (e.g \ac{srd} is stored in directory SICONOS/SRD on CVS server).
\end{itemize} 

After each major release (every tasks of a milestone completed), the version of source code and the corresponding \textbf{updated} documentation must
be tagged by team leaders. The documentation must then be available on \ac{siconos} WP2 web site.

\section{Documents}

The documents concerned by this section are official project documents, such as \ac{srd} or \ac{um}. These documents are written in english and the last 
validated version must be downloadable on wp2-dev collaborative web site. 
After each major update , a document must be validated by a team leader and its version number increased.
The current version of these document is available on CVS project server.
To get more details about Documents writing process, see \ac{cds}.

\section{Code}

% \subsection{Tools}

% CVS, blabla

% Tag, Who When, 

\subsection{Distribution}

%Sources : DEfinition

%Libraries : definition

%etc

The last validated release must be distributed in a tar.gz version. Thanks to the use of GNU-autotools (automake, autoconf), configuration and installation will 
be easy for users. The platform is distributed without external libraries (Lapack, libXml, etc.), which have to be installed by the user on his computer. \\

The general structure of the distribution SICONOS is inspired from GNU standards :

\begin{itemize}
\item directory \textit{bin/} : contains executable files of the platform.
\item directory \textit{lib/} : contains library files 
\item directory \textit{src/} : contains source files. There is a subdirectory for each platform module.
\item directory \textit{sample/} : contains plugin and application examples.
\item directory \textit{doc/} : contains source code documentation.
\end{itemize}
