\section{Purpose of this document}
\label{Sec:ADD-Purpose-Scope}
The \ac{add} is a document which aims to define the software architecture of the \ac{siconos}.
This architecture relies on the specifications of the \ac{srd} and on the external specifications of the \ac{esd}.
The \ac{add} defines and describes the design of the system, in a manner that allows to elaborate it in details progressively. 
It will be used by the Project Manager to share out the detailed conception and programming in work packages.

%% \section{Scope}
%% \label{Sec:ADD-Scope}

%% \begin{ndr}
%% Totalement pipo !!
%% a revoir.
%% \end{ndr}

%% This document relates to only  \ac{engine} and \ac{frontend} of the \ac{siconos} platform.


\section{Scope of the software}
\label{Sec:ADD-Scope}

The software to be produced is a computation platform for \ac{nsds}. It must read the input data of a \ac{nsds}, formalise them
and apply a numerical strategy on them to finally give the results of the calculations.
This document shows the coupling and the cohesion of the platform. The goals are to increase cohesion and
decrease coupling for a better future evolution.

 This document relates only to \acs{engine} and \acs{frontend} of the \ac{siconos} platform. 


% ce que notre soft fera et ne fera pas.
% benefices, objectifs, buts le plus precisement possible
%_______________________________________________

%\section{Definitions, acronyms and abbreviations}
%\label{Sec:ADD-Glossary}
%\begin{itemize}

%\item \ac{Inria} :   Intitut National de Recherche en Informatique et Automatique.
%\item \ac{lmgc} :    Laboratoire de M�canique et de G�nie civil de Montpellier
%\item \ac{matlab} :  "MATrix LABoratory" : computer program for people doing numerical computation, especially linear algebra (matrices).
%\item \ac{NSDS} :    Non-smooth dynamical system.
%\item \ac{scilab} :  scientific software package for numerical computations providing a powerful open computing environment for engineering and                                      scientific applications.
%\item \ac{Siconos} : Modeling, Simulation and control of Non-smooth dynamical systems. 
%\item \ac{SRD} :     Software requirements document.
%\item \ac{WP} :      Work Package.
%\item \ac{XML} .     eXtended Markup Language.
%\item \ac{SA} :        Structured Analysis
%\item \ac{SD} :        Structured Design
%\item \ac{UML} :       Unified Modeling Language
%\item \ac{OCL} :       Object Constraint Language
%\item \ac{ESD} :       External specification document.

%The software to be produced will be composed of several libraries and used as a
%library. At first, it will use numerical computation libraries.




\section{References}
\label{Sec:ADD-References}
\begin{itemize}
        \item \textbf{"Guide to the software architectural design phase",
        PSS-05-04, \ac{esa} 1995}~;
        \item Draft version of \ac{add} of \ac{SICONOS} / \ac{wp}2, January
        2004~;
        \item \ac{srd} of \ac{SICONOS} / \ac{wp}2, March 2004~;
	\item "Cours d'architecture logicielle", Yves Ledru, Master2Pro GI 2004~;
\end{itemize}

\section{Overview}
% decrit ce que le reste du ADD contient
% explique comment le ADD est organis�

%% \begin{ndr}
%% Incomplet
%% a revoir.
%% \end{ndr}

This document presents at first, a global view of the software design. 
Then it explains how this platform will be used presenting its use context. After the description of the design method the architecture of the software is defined. Finally, the components highlighted are described.


%This document will give a system overview, a system context, the conceptual system design and some component description.
%This document is still in progress.   
