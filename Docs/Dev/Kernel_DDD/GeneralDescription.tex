\section{Introduction}
\label{Sec:DDD-Intoduction}
%\section{Purpose of this document}
\label{Sec:SUM-Purpose-Scope}
The purpose of the Software User Manual is to ...


\subsection{Purpose}
\label{Sec:DDD-Purpose}

The \ac{ddd} aims to specify completely the data structures of~\ac{siconos}. This document relies on the global architecture designed in the \ac{add}.
It consists essentially in a class-diagram and a set of constraints on it to eliminate most of its ambiguities. Implementation choices will also be explained and justified here. 


\subsection{Scope}
\label{Sec:DDD-Scope}

This project is a work-package of European project \ac{SICONOS}. \\
Besides the standard features which are required for a software of scientific computing, the objectives of this project are of the following~:
\begin{itemize}
\item To provide a common framework (formalisms and solver tools) for non smooth problems present in various scientific fields : applied Mathematics, Mechanics, Robotics, Electrical networks, etc. 
\item To be able to rely on existing developments. The platform will not re-implement the dedicated tools, which are already used for the modelling of specific systems in various fields, but will provide a framework to their integration.
\item To support exchanges and comparisons of methods between researchers.
\item To disseminate the know-how to other fields of research and industry.
\item To take into account the diversity of users (end users, algorithms developers, framework builders, industrial).
\item To set up standards in terms of modelling of such systems.
\item To ensure software quality by the use of modern software design methods.
\end{itemize}

\subsection{References}
\label{Sec:DDD-References}

\begin{itemize}

\item \textbf{"Guide to the software detailed design and production phase", PSS-05-05, ESA 1995}
\item "Cours de Conduite de Projets Logiciels", Ioannis Parissis, Master 2 Pro GI 2003-2004.
\item "Cours d'Ing\'enirie des mod\`eles", Jean-Marie Favre, Master 2 Pro GI 2003-2004.
\item \ac{add}, April 2004.

\end{itemize}



\subsection{Overview}
\label{Sec:DDD-Overview}

This document specifies in detail the global architecture given in \ac{add}. It contains also development procedures and justifications of implementation choices.


%---------------------------------------------------------------------------%
%\newpage
\section{Project standards, conventions and procedures}
\label{Sec:DDD-ProjectStandards}
%\input{ProjectStandards}

These points are tackled in the \ac{paql}.\\

To summarise it, the main pieces of information are :

\begin{itemize}

\item Design standard~: UML 1.4
\item Documentation standard~: see \ac{adoc}
\item Programming standard~: see \ac{anp}
\item Programming language~: C++
\item Integrated Development environnement~: Eclipse with C++ plugin (CDT).

\end{itemize}
%---------------------------------------------------------------------------%
