\section{XML Schema}
The data of the XML files we can encounter must respect several rules conforming to the \ac{nsds}
and there resolution. Each objects of the platform have specific attributes which are required
or optional. The schema allows to check a lot of rules that are detailed in the \ac{sum}.
To check each information, the schema regroups the defined attributes in several tags relating to
Model, NSDS, DynamicalSystem, Interaction, Relation and NonSmoothLaw, Strategy, TimeDiscretisation,
OneStepIntegrator, OneStepNSProblem.


\section{XML platform}
The management of the XML input/output is made by a set of classes based on the architecture of the
\ac{siconos} platform.

\subsection{Architecture}
In the software, the XML Management uses a tree structure for the XML objects and a DOM tree where all
the data are stored in memory. Each of these objects is linked to the DOM tree, and each object access
only to the data relating to it.

\subsubsection{The XML Management objects}
The figure \ref{fig: Class diagram of the XML management} \& \ref{fig: Class diagram with the links between the platform and the XML part} shows the
structure of the XML Management platform.
The Model is linked to his SiconosModelXML, the NSDS to his NSDSXML, \dots
So, the XML Management platform give to the \ac{siconos} platform the accessors towards the data of the XML DOM tree.
\subsubsection{The data of the XML}
Each XML object can access to the specific branch of the DOM tree which contains the data relating to
the object of the platform. They give an interface to the platform, to manipulate the information of
the DOM tree.
The data stored in the XML Management platform are only used for input and output but never used during the
computations. That's to say the information of the DOM tree are read when the software is launched (if a XML input file is
defined) and data are stored to the DOM tree a the end of each time step of a simulation.
\subsubsection{SiconosDOMTreeTools}
It is a toolbox to manipulate the data of the platform to store them under XML format.


\subsection{Unfolding of the creation of the platform}
The two ways to construct the platform are using similar mechanisms, and especially the same creating
method.
\subsubsection{XML file loading}
At first, when a XML file is loaded, the data of the file are copied in memory in a DOM tree. From
there, the XML Management platform is built.\\
The SiconosModelXML owns the DOM tree and create NSDSXML and StrategyXML objects. The created objects
only know the branch of the DOM tree relating to them. Gradually, the NSDSXML will create the
different XML objects of the dynamical systems (DSXML, LagrangianNLDSXML, LagrangianTIDSXML,
LinearSystemDSXML), and the different interactions.\\
Then, after all the XML objects have been created, the \ac{siconos} platform is built.\\
The Model which has lead the construction of the XML platform begin the creation of the NSDS,
DynamicalSystem, LagrangianNLDS, ..., Interaction, Relation, ..., NonSmoothLaw, ..., Strategy,
TimeDiscretisation, OneStepIntegrator, Moreau, ..., OneStepNSProblem, LCP and QP by using the
relating XML objects.\\
The construction of each object of the platform is made by calling a
"createXxxxx" method (createModel(...), createNSDS(...), createStrategy(...), ...). One parameter
corresponding to the XML object is enough to give the right data to the platform's object for his
construction.
\subsubsection{Creating the platform without XML input file}
Another way to build the platform is to do it manually.
To do this, each object of the \ac{siconos} platform owns functions designed to create/add the
platform's objects belonging to it. These functions must have in parameters all the required data for
the C++ objects. Here are these methods :
\begin{itemize}
	\item In the Model class :
	\begin{itemize}
		\item createNSDS(attributes required for a NSDS : bool BVP)
		\item createTimeStepping()
		\item createEvenrDriven()
	\end{itemize}
	
	\item In the NSDS class :
	\begin{itemize}
		\item addNonLinearSystemDS(number, n, x0, BasicPlugin:vectorField)
		\item addLinearSystemDS(number, n, x0)
		\item addLagrangianNLDS(number, ndof, q0, velocity0, BasicPlugin:computeMass,
		BasicPlugin:computeFInt, BasicPlugin:computeFExt,
		BasicPlugin:computeJacobianQFInt, BasicPlugin:computeJacobianVelocityFInt,
		BasicPlugin:computeJacobianQQNLInertia,
		BasicPlugin:computeJacobianVelocityQNLInertia, BasicPlugin:computeQNLInertia)
		\item addLagrangianTIDS(number, ndof, q0, velocity0, BasicPlugin:computeMass, BasicPlugin:computeFExt, K, C)
		\item addInteraction(number)
	\end{itemize}
	
	\item In the DynamicalSystem class :
	\begin{itemize}
		\item createLinearBC()
		\item createNLinearBC()
		\item createPeriodicBC()
	\end{itemize}
	
	\item In the Interaction class :
	\begin{itemize}
		\item createLagrangianLinearR()
		\item createLagrangianNonLinearR()
		\item createLinearTIR()
		\item createNewtonImpactLawNSL()
		\item createComplementaryConditionNSL()
		\item createRelayNSL()
	\end{itemize}
	
	\item In the Strategy class :
	\begin{itemize}
		\item createTimeDiscretisation()
		\item addMoreau()
		\item addLsodar()
		\item addAdams()
		\item addLCP()
		\item addQP()
	\end{itemize}
	
\end{itemize}
All these functions are calling the createXxxx(...) function of the object to create.

\subsection{Saving the data of the platform}
The save of the platform's data is lead from the Model. The function which do this job is
"saveToXMLFile". It has several things to do before saving tha data in a XML file :
\begin{itemize}
	\item checkXMLPlatform() : This first function will perfom verifications on the XML Management platform. It checks the
	link between the platform's objects and the XML Management objects. If the XML Management
	platform doesn't exist, it will be created and linked to the objects of the \ac{siconos}
	platform. Otherwise, every link between the platform and the XML Management is checked to ensure
	the availability of the XML platform objects.
	\item savePlatformToXML() : Now, all the objects of the platform are linked to their XML Management object. Therefore,
	it is possible to save the data of the platform to the XML DOM tree. The information
	contained in the platform are saved in the DOM tree by using the specific functions given
	by the XML object.
	\item checkXMLDOMTree() : The data of the DOM tree is up to date. But It is important to check that these data still
	respect the XML schema. 
	\item saveSiconosModelInXMLFile(xmlFile) : The last action to be done is to write the data in memory to a file.
\end{itemize}
