
\section{Configuration and installation management}
\label{sec:configuration}

This section is dedicated to the packaging solution chosen for the project.

\subsection{Autotools}
\label{sec:autotools}

Autotools is a generic name for mainly three GNU tools : Autoconf, Automake and Libtool. 

To explain briefly the role of each of these tools, we can cite the introduction of GNU autoconf, Automake and Libtool by Gary V. Vaughan, Ben Elliston, Tom Tromey  and Ian Lance Taylor (online version : http://sources.redhat.com/autobook/autobook/autobook\_toc.html): \\

\textit{Autoconf, Automake and Libtool are packages for making your software more portable and to simplify building it--usually on someone else's system.} \\

\textit{Autoconf is a tool that makes your packages more portable by performing tests to discover system characteristics before the package is compiled. Your source code can then adapt to these differences. } \\

\textit{Automake is a tool for generating Makefile's--descriptions of what to build--that conform to a number of standards. Automake substantially simplifies the process of describing the organization of a package and performs additional functions such as dependency tracking between source files. } \\

\textit{Libtool is a command line interface to the compiler and linker that makes it easy to portably generate static and shared libraries, regardless of the platform it is running on.}


\subsubsection{configure script}
\label{sec:configure}
The file configure.ac (in Kernel directory) is the input file for autoconf. We specify in this file each test concerning the hardware and software configuration of the computer where we want to install the platform. This file uses m4 macro language. Briefly, the tests are :
\begin{itemize}
\item C++ compiler.
\item Classical shell commands : cp, rm, awk...
\item Standard system header files.
\item Installation of required external libraries.
\end{itemize}

configure.ac references too each Makefiles of the platform, which are generated from automake input files (named Makefile.am).

Autoconf takes this file in input and creates configure script. When this script is executed, it produces Makefiles and sets compiler flags, library paths, etc.


\subsubsection{Automake input files}
We do not directly write the makefiles of the platform. We define input files named Makefile.am which are processed by automake and autoconf. In a Makefile.am file, you can wrote targets as in a classical makefile, but you can also use more advanced features. Just look at this exmaple :\\

\begin{verbatim}
bin_PROGRAMS = doit
doit_SOURCES = doit.c main.c
\end{verbatim}

These 2 lines indicate that we want to build a program named doit, which is composed of doit.c and main.c source files. We have not to specify compilation rules and flags, this is automatically done thanks to configure script result data (C compiler type, etc.). This program will be installed by typing �make install� in the bin directory of the installation location specified during configure script execution. \\

As you can see, these files are easier to understand and to maintain than a classical Makefile.


\subsubsection{Libtool}
\label{sec:libtool}


Libtool manages the creation of the platform libraries. To create a library, we just have to use some automake macros like this :

\begin{verbatim}
lib_LTLIBRARIES     = libshell.la
libshell_la_SOURCES = object.c subr.c symbol.c
\end{verbatim}

libshell.la is a libtool library. from this file, libtool can produce a static library and a shared library. These libraries will be installed by typing �make install� in the lib directory of the installation location specified during configure script execution.



\subsection{Distribution}
\label{sec:distribution}






% Two deliverables will be distributed, corresponding to the two main uses of the platform~: a basic one for end-users, with only binaries, and a complete one with source files, to allow users to develop news functionalities for the software.

% The file names and repertories of the platform core are written in small letters whereas user files have only their first letter in capitals. \\

% \section{Expert users distribution}
% \label{dudev}

% The software \textit{siconos} will be a set of C++ libraries driven by a main program. This kind of libraries are named in the following as "internal libraries" (i.e. libraries developed in \ac{siconos}, in opposition to the "external libraries" which are for example libXML2 or Lapack++). \\

% In this distribution, the root directory is named \textit{SICONOS/}. It contains the following sub-directories and files : 
% \begin{itemize}

% \item  \textit{config/} : contains the configuration files needed by the platform. The sub-dir \textit{xmlschema} holds schema of \ac{xml} files.

% \item  \textit{doc/} : contains the \ac{um} in a pdf file \textit{siconos.pdf}. The sub-dir \textit{api} holds the documentation of the internal libraries in html pages generated by Doxygen.

% \item \textit{ext/} : contains external libraries (like libXML2) needed by the platform (if the user has not already installed them on his computer).

% \item \textit{include/} : contains header files of internal libraries.

% \item \textit{lib/} : contains the internal libraries. If a library XXXX is dynamic, the file corresponding is libXXXX.so. If a library XXXX is static, the file corresponding is libXXXX.a.

% \item \textit{sample/} : examples of uses of the platform.

% \item \textit{siconos/} : contains the binary file \textit{siconos} and its source file \textit{siconos.cpp}. This binary permits to launch a simulation.

% \item \textit{src/} : contains the source files of the internal libraries classified by module. For each module, there is a directory, with a Makefile to compile sources and tests. The test sources and all data needed by test are placed in a subdirectory \textit{test}.
	
% %	\begin{itemize}	
% %	\item \textit{bin/} : binaries of the module and of its tests ;
% %	\item \textit{doc/} : particular documentation of the module if need ;
% %	\item \textit{test/} : source files of the unitary tests of the module;		
% %	\end{itemize}

% %There can be some other sub-directories if the module needs it. For example, the managing simulation data module with \ac{xml} files needs a subdir \textit{systemsschema/} to store several \ac{xsd} used to verify the conformity of \ac{xml} files given by users to realize a simulation. 

% \item \textit{Makefile or shell script} : builds the application (libraries, binaries, etc.) and generates the documentation. It must be able to build entirely the software from the data of the src/ directory only.

% \item \textit{README.txt} : instructions to install the software, etc.

% \end{itemize}

% %See figure \ref{archiDev} which illustrates this files organisation.

% %\begin{figure}[!hbp]
% %\begin{center}
% %\includegraphics[width= \textwidth]{figure/archiDev.pstex}
% %\caption{Expert user distribution}
% %\label{archiDev}
% %\end{center}
% %\end{figure}



% \section{End user distribution}

% This distribution is the minimal one and contains only the basic functionalities to run a simple simulation.It is the same than the expert-user deliverable without development directories : src/ and the part of doc/ concerning implementation.

% %\begin{figure}[!hbp]
% %\begin{center}
% %\includegraphics[ width= \textwidth]{figure/archiUtil.pstex}
% %\caption{End user distribution}
% %\label{archiUtil}
% %\end{center}
% %\end{figure}

% \section{User data}
% Each system which is simulated by the platform must have a particular workspace, which is a directory. The path of the directory should be given by the user. By default, simulations are placed in the samples directory of the platform. \\
% A directory of a simulation contains two subdirectories :
% \begin{itemize}
% \item \textit{input/ }: holds \ac{xml} files given by the user (initial data for example) and possibly a command C++ file to drive the simulation.
% \item \textit{output/ }: holds several files saved by the platform during the simulation (complete state of the system in \ac{xml} files, matrix files, particular of the system, etc.). 
% \end{itemize}	



