%\documentclass[a4paper,twoside,openright,makeidx,12pt]{book}

%%$Id: macro.tex,v 1.10 2004/12/08 13:38:58 acary Exp $


%\usepackage{a4wide}
\textheight 25cm
\textwidth 16.5cm
\topmargin -1cm
%\evensidemargin 0cm
\oddsidemargin 0cm
\evensidemargin0cm
\usepackage{layout}


\usepackage{amsmath}
\usepackage{amssymb}
\usepackage{minitoc}
%\usepackage{glosstex}
\usepackage{colortbl}
\usepackage{hhline}
\usepackage{longtable}

%\usepackage{glosstex}
%\def\glossaryname{Glossary of Notation}
\def\listacronymname{Acronyms}

\usepackage[outerbars]{changebar}\setcounter{changebargrey}{20}
%\glxitemorderdefault{acr}{l}

%\usepackage{color}
\usepackage{graphicx,epsfig}
\graphicspath{{figure/}}
\usepackage[T1]{fontenc}
\usepackage{rotating}

%\usepackage{algorithmic}
%\usepackage{algorithm}
\usepackage{ntheorem}
\usepackage{natbib}


%\renewcommand{\baselinestretch}{2.0}
\setcounter{tocdepth}{2}     % Dans la table des matieres
\setcounter{secnumdepth}{3}  % Avec un numero.



\newtheorem{definition}{Definition}
\newtheorem{lemma}{Lemma}
\newtheorem{claim}{Claim}
\newtheorem{remark}{Remark}
\newtheorem{assumption}{Assumption}
\newtheorem{example}{Example}
\newtheorem{conjecture}{Conjecture}
\newtheorem{corollary}{Corollary}
\newtheorem{OP}{OP}
\newtheorem{problem}{Problem}
\newtheorem{theorem}{Theorem}


\newcommand{\CC}{\mbox{\rm $~\vrule height6.6pt width0.5pt depth0.25pt\!\!$C}}
\newcommand{\ZZ}{\mbox{\rm \lower0.3pt\hbox{$\angle\!\!\!$}Z}}
\newcommand{\RR}{\mbox{\rm $I\!\!R$}}
\newcommand{\NN}{\mbox{\rm $I\!\!N$}}

\newcommand{\Mnn}{\mathcal M^{n\times n}}
\newcommand{\Mnp}[2]{\ensuremath{\mathcal M^{#1\times #2}}}



\newcommand{\Frac}[2]{\displaystyle \frac{#1}{#2}}

\newcommand{\DP}[2]{\displaystyle \frac{\partial {#1}}{\partial {#2}}}

% c++ variables writting
\newcommand{\varcpp}[1]{\textit{#1}}
% itemize
\newcommand{\bei}{\begin{itemize}}
\newcommand{\ei}{\end{itemize}}

\newcommand{\ie}{i.e.}
\newcommand{\eg}{e.g.}
\newcommand{\cf}{c.f.}
\newcommand{\putidx}[1]{\index{#1}\textit{#1}}

\def\Er{{\rm I\! R}}
\def\En{{\rm I\! N}} 
\def\Ec{{\rm I\! C}}
 
\def\zc{\hat{z}}
\def\wc{\hat{w}}

\font\tete=cmr8 at 8 pt
\font\titre= cmr12 at 20 pt 
\font\titregras=cmbx12 at 20 pt

%----------------------------------------------------------------------
%                  Modification des subsubsections
%----------------------------------------------------------------------
\makeatletter
\renewcommand\thesubsubsection{\thesubsection.\@alph\c@subsubsection}
\makeatother

%----------------------------------------------------------------------
%             Redaction note environnement
%----------------------------------------------------------------------
\makeatletter
\theoremheaderfont{\scshape}
\theoremstyle{marginbreak}
\theorembodyfont{\upshape}
%\newtheorem{rque}{\bf Remarque}[chapter]
%\newtheorem{rque1}{\bf \fsc{Remarque}}[chapter] !!! \fsc est une commande french
\newtheorem{ndr1}{\textbf{\textsc{Redaction note}}}[section]

\newenvironment{ndr}%
{%
\tt
%\centerline{---oOo---}
\noindent\begin{ndr1}%
}%
{%
\begin{flushright}%
%\vspace{-1.5em}\ding{111}
\end{flushright}%
\end{ndr1}%
%\centerline{---oOo---}
}

\makeatother

%----------------------------------------------------------------------
%             Redaction note environnement V.ACARY
%----------------------------------------------------------------------
\makeatletter
\theoremheaderfont{\scshape}
\theoremstyle{marginbreak}
\theorembodyfont{\upshape}
%\newtheorem{rque}{\bf Remarque}[chapter]
%\newtheorem{rque1}{\bf \fsc{Remarque}}[chapter] !!! \fsc est une commande french
\newtheorem{ndr1va}{\textbf{\textsc{Redaction note V. ACARY}}}[section]

\newenvironment{ndrva}%
{%
\tt
%\centerline{---oOo---}
\noindent\begin{ndr1va}%
}%
{%
\begin{flushright}%
%\vspace{-1.5em}\ding{111}
\end{flushright}%
\end{ndr1va}%
%\centerline{---oOo---}
}

\makeatother
%----------------------------------------------------------------------
%             Redaction note environnement V.ACARY
%----------------------------------------------------------------------
\makeatletter
\theoremheaderfont{\scshape}
\theoremstyle{marginbreak}
\theorembodyfont{\upshape}
%\newtheorem{rque}{\bf Remarque}[chapter]
%\newtheorem{rque1}{\bf \fsc{Remarque}}[chapter] !!! \fsc est une commande french
\newtheorem{ndr1fp}{\textbf{\textsc{Redaction note F. PERIGNON}}}[section]

\newenvironment{ndrfp}%
{%
\tt
%\centerline{---oOo---}
\noindent\begin{ndr1fp}%
}%
{%
\begin{flushright}%
%\vspace{-1.5em}\ding{111}
\end{flushright}%
\end{ndr1fp}%
%\centerline{---oOo---}
}

\makeatother
%----------------------------------------------------------------------
%                  Chapter head enviroment
%----------------------------------------------------------------------
\newenvironment{chapter_head}
{%
\begin{center}%
-------------------- oOo --------------------\\%
\ \\%
\begin{minipage}[]{14cm}%
\noindent\normalsize\advance\baselineskip-1pt %
}%
{%
\par\end{minipage}%
\ \\%
\ \\%
-------------------- oOo --------------------
\end{center}%
\vspace*{\stretch{1}}%
\clearpage%
\thispagestyle{empty}%
\vspace*{\stretch{1}}%
\minitoc%
\vspace*{\stretch{2}}%
\clearpage%
}

%%% Local Variables: 
%%% mode: latex
%%% TeX-master: "report"
%%% End: 


%\begin{document}
%\pagestyle{empty}
%\renewcommand{\arraystretch}{1.8}



Two deliverables will be distributed, corresponding to the two main uses of the platform~: a basic one for end-users, with only binaries, and a complete one with source files, to allow users to develop news functionalities for the software.

The file names and repertories of the platform core are written in small letters whereas user files have only their first letter in capitals. \\

\section{Expert users distribution}
\label{dudev}

The software \textit{siconos} will be a set of C++ libraries driven by a main program. This kind of libraries are named in the following as "internal libraries" (i.e. libraries developed in \ac{siconos}, in opposition to the "external libraries" which are for example libXML2 or Lapack++). \\

In this distribution, the root directory is named \textit{SICONOS/}. It contains the following sub-directories and files : 
\begin{itemize}

\item  \textit{config/} : contains the configuration files needed by the platform. The sub-dir \textit{xmlschema} holds schema of \ac{xml} files.

\item  \textit{doc/} : contains the \ac{um} in a pdf file \textit{siconos.pdf}. The sub-dir \textit{api} holds the documentation of the internal libraries in html pages generated by Doxygen.

\item \textit{ext/} : contains external libraries (like libXML2) needed by the platform (if the user has not already installed them on his computer).

\item \textit{include/} : contains header files of internal libraries.

\item \textit{lib/} : contains the internal libraries. If a library XXXX is dynamic, the file corresponding is libXXXX.so. If a library XXXX is static, the file corresponding is libXXXX.a.

\item \textit{sample/} : examples of uses of the platform.

\item \textit{siconos/} : contains the binary file \textit{siconos} and its source file \textit{siconos.cpp}. This binary permits to launch a simulation.

\item \textit{src/} : contains the source files of the internal libraries classified by module. For each module, there is a directory, with a Makefile to compile sources and tests. The test sources and all data needed by test are placed in a subdirectory \textit{test}.
	
%	\begin{itemize}	
%	\item \textit{bin/} : binaries of the module and of its tests ;
%	\item \textit{doc/} : particular documentation of the module if need ;
%	\item \textit{test/} : source files of the unitary tests of the module;		
%	\end{itemize}

%There can be some other sub-directories if the module needs it. For example, the managing simulation data module with \ac{xml} files needs a subdir \textit{systemsschema/} to store several \ac{xsd} used to verify the conformity of \ac{xml} files given by users to realize a simulation. 

\item \textit{Makefile or shell script} : builds the application (libraries, binaries, etc.) and generates the documentation. It must be able to build entirely the software from the data of the src/ directory only.

\item \textit{README.txt} : instructions to install the software, etc.

\end{itemize}

%See figure \ref{archiDev} which illustrates this files organisation.

%\begin{figure}[!hbp]
%\begin{center}
%\includegraphics[width= \textwidth]{figure/archiDev.pstex}
%\caption{Expert user distribution}
%\label{archiDev}
%\end{center}
%\end{figure}



\section{End user distribution}

This distribution is the minimal one and contains only the basic functionalities to run a simple simulation.It is the same than the expert-user deliverable without development directories : src/ and the part of doc/ concerning implementation.

%\begin{figure}[!hbp]
%\begin{center}
%\includegraphics[ width= \textwidth]{figure/archiUtil.pstex}
%\caption{End user distribution}
%\label{archiUtil}
%\end{center}
%\end{figure}

\section{User data}
Each system which is simulated by the platform must have a particular workspace, which is a directory. The path of the directory should be given by the user. By default, simulations are placed in the samples directory of the platform. \\
A directory of a simulation contains two subdirectories :
\begin{itemize}
\item \textit{input/ }: holds \ac{xml} files given by the user (initial data for example) and possibly a command C++ file to drive the simulation.
\item \textit{output/ }: holds several files saved by the platform during the simulation (complete state of the system in \ac{xml} files, matrix files, particular of the system, etc.). 
\end{itemize}	



