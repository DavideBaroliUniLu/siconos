\section{Context and design of the system}
\ac{numerics} will be used to make specific computations on non-smooth dynamical systems.\\

It will be written in C and used as a library. But it will also use methods already written in Fortran.
It's planned to be used through a the \ac{siconos} platform as a computation library, or as a stand-alone library.

It may be decomposed in several parts according to future extensions~:
  \begin{itemize}
  	\item \ac{lapack}\\
		Routines for solving systems of simultaneous linear equations.
    \item NSS pack\\
		Non smooth solver pack.
    \item \ac{ode} pack\\
		Initial value problem solving for ordinary differential equation systems.
	\item future extensions...\\
  \end{itemize}
  
  
%\section{Costs and benefits of the architecture}

\section{Prototyping exercises}
No prototypes are necessary to evaluate the architecture. There is no difficulty about the software, no technologies that have to be tested. \ac{numerics} is mainly know-how that physicists bring to the library.


\section{External interfaces}
The library will offer computation methods through interfaces encapsulating each used internal library.\\
Each module such as NSS pack, \ac{ode} pack, ... have to be encapsulated to give a C/C++ \ac{api} to unify the interfaces and to grant the access to these functions to \ac{siconos} platform.
