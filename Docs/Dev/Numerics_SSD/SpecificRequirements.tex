%---------------------------------------------------------------------------%
%\section{Model and numerical methods related requirements}

%---------------------------------------------------------------------------%
\section{Functional requirements}
\label{Sec:SSD-FunctionalRequirements}
%\subsection{\ac{siconos}/Numerics}
\begin{longtable}{%
    |>{\columncolor[gray]{.8}}p{0.1\textwidth}%
    |>{\columncolor[gray]{.95}}p{0.9\textwidth}|}
  \hline
  \rowcolor[gray]{.8}   SR. Id. & Software requirements description \\
  \hline 
  \hline
  & \textbf{General  requirements (\ac{siconos}/Numerics)}\\
  \hline 
  F.1.001 & The module \ac{siconos}/Numerics must provide basic algebra objects (vector, matrices, quaternions) \bf{ \sf relying on a matrix template library} \\
  F.1.002 & The module \ac{siconos}/Numerics must provide  high performance methods for  basic vector and matrix operations\\
  F.1.003 & The module \ac{siconos}/Numerics must support various storage methods for matrices:
  \begin{enumerate}
  \item Dense(full)
  \item Band
%fd sans interet  \item Skyline
  \item Sparse
  \end{enumerate}\\
 \hline 
 & \textbf{ Numerical functionalities  requirements (\ac{siconos}/Numerics)}\\
  \hline
  F.1.010 & The module  \ac{siconos}/Numerics has to perform basic linear algebra computations  relying on standard libraries for numerical computation (e.g. LAPACK):
  \begin{enumerate}
  \item Solution of linear system 
  \item Eigenvalue and eigenvectors problem
  \item Singular Value Decomposition
  \end{enumerate}
  \\
  F.1.011 & The module \ac{siconos}/Numerics has to perform solution for mathematical programming problems:
  \begin{enumerate}
  \item Linear Complementarity Problem (LCP) (direct and iterative solutions)
  \item Relay
  \item Quadratic problem (QP)
  \item Non Linear Complementarity Problem (NCP)
    \begin{enumerate}
      \item Implicit interaction problem (for example 3D friction) 
      \item Non linear interaction problem (for example contact and friction)
      \item Non linear smooth problem
      \item A mix of the previous cases 
    \end{enumerate}
  \item Root finding for non smooth (generalized) equations
  \item General non smooth optimization problem 
  \end{enumerate} \\
  F.1.012 & The module \ac{siconos}/Numerics has to perform basic computation for smooth time integration (one-step integration)  \\
  F.1.013 & The module \ac{siconos}/Numerics has to perform root finding for non linear smooth equations (Newton's method)\\
  F.1.014 & The module \ac{siconos}/Numerics has to perform numerical, analytical and automatic differentiation\\
  \hline

  & \textbf{Interface  requirements (\ac{siconos}/Numerics)}\\
  \hline 
   F.1.020 & The module \ac{siconos}/Numerics must provide a common interface to various methods dedicated to one specific type of numerical problem\\
  \hline
  \caption{\ac{siconos}/Engine. Software Requirements}\\
\end{longtable}


 
%%% Local Variables: 
%%% mode: latex
%%% TeX-master: "../report"
%%% End: 



%---------------------------------------------------------------------------%

\section{Performance requirements}
\begin{longtable}{%
|>{\columncolor[gray]{.8}}p{0.1\textwidth}%
|>{\columncolor[gray]{.95}}p{0.9\textwidth}|}
   \hline
\rowcolor[gray]{.8}   SR. Id. & Software requirements description \\
      \hline 
   & \textbf{ Performance requirements }\\
   \hline
   PER.00 & The software mustn't be more than 10\% slower than \ac{lmgc90} for the same treatement on the same computer. \\
\hline
\caption{Performance Requirements}\
\end{longtable}

%---------------------------------------------------------------------------%
%\section{Data representation requirements}


%\begin{longtable}{% |>{\columncolor[gray]{.8}}p{0.1\textwidth}% |>{\columncolor[gray]{.95}}p{0.9\textwidth}|}
%\hline
%\rowcolor[gray]{.8}   SR. Id. & Software requirements description \\
%   \hline 
%   & \textbf{Data representation }\\
%   \hline
%   DAT.00 & Files representing formalized \ac{nsds} and generated by the s/w will be \ac{xml} files. \\  
%\hline
%\caption{Data representation requirements}
%\end{longtable}
%---------------------------------------------------------------------------%


%---------------------------------------------------------------------------%
\section{Interface requirements and Users environments related requirements}
\begin{longtable}{%
    |>{\columncolor[gray]{.8}}p{0.1\textwidth}%
    |>{\columncolor[gray]{.95}}p{0.9\textwidth}|}
  \hline
  \rowcolor[gray]{.8}   SR. Id. & Software requirements description \\
  \hline 
  & \textbf{Users interfaces requirements }\\
  \hline
  INT.00 & The software should provide an API. \\
  INT.01 & This API can be used by a C++ software.\\
  \hline 
%  & \textbf{Interfaces requirements with Existing modelling software }\\ 
%  \hline
%  INT.10 & High-level functions of the software must be usable through \ac{scilab}.  \\
%  INT.11 & High-level functions of the software must be usable through \ac{matlab}.  \\
%  \hline
  \caption{Interface requirements}\\
\end{longtable}


Discuss the use conditions (user-friendly, accessibility, \ldots) and the level of abstraction for different classes of users :
\begin{itemize}
%\item frameworks builders
%\item Component builders
\item Algorithm developers
\item End users
\end{itemize}

%---------------------------------------------------------------------------%
\section{Resource requirements}
\begin{longtable}{%
|>{\columncolor[gray]{.8}}p{0.1\textwidth}%
|>{\columncolor[gray]{.95}}p{0.9\textwidth}|}
   \hline
\rowcolor[gray]{.8}   SR. Id. & Software requirements description \\
      \hline 
   & \textbf{Resource requirements }\\
   \hline
   RES.00 &  This configuration is recommended : processor 800 MHz, 512 Mo Ram, 20 Mo of free space on HDD.\\
\hline
\caption{Resource requirements}
\end{longtable}
%---------------------------------------------------------------------------%

\section{Documentation requirements}
\begin{longtable}{%
    |>{\columncolor[gray]{.8}}p{0.1\textwidth}%
    |>{\columncolor[gray]{.95}}p{0.9\textwidth}|}
  \hline
  \rowcolor[gray]{.8}   SR. Id. & Software requirements description \\
  \hline 
  & \textbf{ Documentation type requirements }\\
  \hline
  DOC.1.00 & The documentation must contain a Software User Manual (SUM)  \\
  DOC.1.01 & The documentation must contain a Tutorial Manual\\
  DOC.1.02 & The documentation must contain a Example problems Manual\\
  DOC.1.03 & The documentation must contain a Benchmarks \& Verification Manual\\
  DOC.1.04 & The documentation must contain a Theory Manual\\
  DOC.1.05 & The documentation must contain a Interface users Manual\\
  DOC.1.06 & The documentation must contain a Developer Manual\\
  \hline 
  & \textbf{ Documentation format requirements }\\
  \hline
  DOC.2.00 & The source of the various manual must be in TeX format\\
  DOC.2.01 & The various manual must be available in HTML format\\
  DOC.2.02 & The various manual must be available in PDF format\\
  DOC.2.03 & The various manual must be written in English.\\
 \hline
  \caption{Documentation requirements}\\
\end{longtable}



%---------------------------------------------------------------------------%
\section{Portability requirements}
\begin{longtable}{%
|>{\columncolor[gray]{.8}}p{0.1\textwidth}%
|>{\columncolor[gray]{.95}}p{0.9\textwidth}|}
   \hline
\rowcolor[gray]{.8}   SR. Id. & Software requirements description \\
      \hline 
   & \textbf{  Hardware and Operating system support }\\
   \hline
   POR.1.00 & The s/w must run on PC/Linux environment \textit{kernel 2.4.20-8}\\
   POR.1.01 & The s/w must run on PC/Windows environment \textit{Microsoft Windows 2000}\\
   POR.1.02 & The s/w must run on Sun Workstation/Sun OS-Solaris environment \textit{Solaris 5.8}\\
   POR.1.03 & The s/w must run on  Apple/Mac OS X environment \\
   \hline 
   & \textbf{ Data portability}\\
   \hline
   POR.2.00 & The ASCII data files must be portable (specify an unique charset)\\ 
   POR.2.01 & The BINARY data files must be portable (specify a binary format,IEEE norm)\\ 
   \hline
\caption{Portability requirements}\\
\end{longtable}

%---------------------------------------------------------------------------%
%\section{Quality requirements}

%After an analysis of \ac{siconos} software requirements, four priority software quality factors (Mc Call criteria) have been chosen. Each one has a mark of priority.
%\begin{longtable}{%
%|>{\columncolor[gray]{.8}}p{0.1\textwidth}%
%|>{\columncolor[gray]{.95}}p{0.9\textwidth}|}
%   \hline
%\rowcolor[gray]{.8}   SR. Id. & Software requirements description \\
%      \hline 
%   & \textbf{ Quality requirements }\\
%   \hline
%   QUA.00 & Adaptability (or flexibility). \textit{9/10}\\
%   QUA.01 & Efficiency.\textit{8/10}\\
%   QUA.02 & Couplability. \textit{7/10}\\ 
%   QUA.03 & Portability. \textit{5/10}\\
%   \hline
%\caption{ Quality requirements}\\
%\hline
%\end{longtable}

%Here are short definitions of those criteria and the reasons why they were chosen as priority quality factors for this project.

%\begin{itemize}
%\item   Adaptability : \\
%        It measures the ability of the software to make easier additions of new functionalities or modifications, even suppression of existing functionalities. This factor has been chosen because users must be able to add easily new algorithms or computations libraries. 
%\item   Efficiency : \\
%        It measures the ability of the software to minimize its resources needs (cpu time, memory, etc.).
%        This criterion is necessary for a scientific computation software.
%\item   couplability : \\
%        Capacity of the software to be used with other softwares (data exchanges, calls, ...).
%        Like the plateforme can be accessible via \ac{scilab}, this factor is important.
%\item   Portability : \\
%        Capacity of the software to reduce consequences of an environment technical change (hardware or software).
%        Users come from several laboratories and companies, and work under several OS.
%\end{itemize}
