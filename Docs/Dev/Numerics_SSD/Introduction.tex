%$Id: Introduction.tex,v 1.6 2004/12/22 13:51:30 jbarbier Exp $
\section{Purpose of this document}
\label{Sec:SSD-Purpose-Scope}
The purpose of the Software Specification Document Document is to define the user and software requirements, and the architecture design of \ac{siconos}/Numerics.
The \ac{ssd} is a contractual document which aims to define precisely the software to realize. It describes functionalities and characteristics of the product and constraints of development and exploitation. It is addressed to users and software framework builders. \\
\\
This document will be used as basis : 
\begin{itemize}
\item For the evaluation of the final product,
\item For the editing of test plan,
\item For the editing of the \ac{ddd}.
\end{itemize}


\section{Context of \ac{numerics}}
\label{Sec:SSD-Context}
\ac{numerics} is toolbox for computation relating to non-smooth dynamical systems. The main use of its capabilities is in the \ac{siconos} platform.

%\subsection{Exploitation context}


\section{References}
\label{Sec:SSD-References}
\begin{itemize}
\item \textbf{"Guide to applying the software engineering standards to small software projects", BSSC(96)2 Issue1, ESA 1996}
\item \ac{tm}
\item \ac{SICONOS} Contract Number IST-2001-37172 and annexes

\end{itemize}




\section{Overview}
This software specification is composed of two phases :
\begin{enumerate}
\item The specification of the requirements
	\begin{itemize}
	\item A general description of the project. This part explains the purpose of this project and its integration into European project \ac{SICONOS}.
	\item The problem definition phase. The scope of the software must be defined. Specific user requirements must be identified and documented. The section has to provide a general description of what the user expects the software to do. It should state the specific user requirements as clearly and consistently as possible. 
	\end{itemize}
\item The architecture design\\
	It defines and describes the design of the system, in a manner that allows to elaborate it in details progressively. 
\end{enumerate}


