Two deliverables can be distributed, corresponding to the two main uses of the platform~: a basic one for end-users, with only binaries, and a complete one with source files, to allow users to develop news functionalities for the software.
Here we describe the architecture of the distribution for end users. Developers can refer to the \ac{ddd} for complete architecture.

\section{Files and directories}
The software \textit{siconos} will be a set of C++ libraries driven by a main program. This kind of libraries are named in the following as "internal libraries" (i.e. libraries developed in \ac{siconos}, in opposition to the "external libraries" which are for example libXML2 or Lapack++). \\

In this distribution, the root directory is named \textit{SICONOS/}. It contains the following sub-directories and files~: 
\begin{itemize}

\item  \textit{bin/} : contains binaries of the platform.

\item \textit{include/} : contains header files of internal libraries.

\item \textit{lib/} : contains the libraries of the platform.

\item \textit{share/} : contains some configuration files, necessary ton run a simulation with the platform : \acs{xml} schema, ...

\end{itemize}


%\section{User data}
%Each system which is simulated by the platform must have a particular workspace, which is a directory. The path of the directory should be given by the user. By default, simulations files (\ac{xml} files and vector / matrix data files) are placed in the \textit{samples} directory of the platform.
