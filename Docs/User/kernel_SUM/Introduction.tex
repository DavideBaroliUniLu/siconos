\section{Intended Readership}
Two main types of users have been identified during the software requirements analysis phase~: end users and developers (see \ac{srd}). \\

Typically, end-users are scientists who use the platform to make simulations of \ac{nsds}. They have knowledge about mechanics or electronics, but don't need to be expert in computer sciences. On the contrary, platform and plugins developers have to be skilled in object oriented languages and \ac{xml} files format.

\section{Applicability Statements}
This document apply to the first version of \ac{siconos}.

\section{Purpose}
This project is a work-package of European project \ac{SICONOS}. \\
Besides the standard features which are required for a software of scientific computing, the objectives of this project are of the following~:
\begin{itemize}
\item To provide a common framework (formalisms and solver tools) for non smooth problems present in various scientific fields~: applied Mathematics, Mechanics, Robotics, Electrical networks, etc. 
\item To be able to rely on existing developments. The platform will not re-implement the dedicated tools, which are already used for the modelling of specific systems in various fields,but will provide a framework to their integration.
\item To support exchanges and comparisons of methods between researchers.
\item To disseminate the know-how to other fields of research and industry.
\item To take into account the diversity of users (end users, algorithms developers, framework builders, industrial).
\item To set up standards in terms of modelling of such systems.
\item To ensure software quality by the use of modern software design methods.
\end{itemize}

This manual has to be a reference document concerning the using of the \ac{SICONOS} platform. For the end-users, it provides tutorials and basic examples about all the functionalities offered by the platform to realize a simulation of \ac{nsds}. For developers, the manual details the commands of the platform, how to create, add and test a plugin. It references sections of other documents of \ac{SICONOS} which concern data structures, implementation choices, etc.


\section{How to use this document}
The document explain first the platform architecture. This section describe all the directory and their contents provide with the platform.
After, this document explain how to install and execute the platform.
In the next section, an explication of the \ac{siconos} \ac{xml} file format and how to write it is given.
Finally, a description of the possible errors is given 

%\section{Related documents}
%\begin{ndr}
%siconos documents which could be useful for the user. Relations between these documents.
%\end{ndr}

\section{Conventions}
Convention to show a display on the screen~: \\

\begin{minipage}{\textwidth}
\small{\textsf{\% make build}}\\
\small{\textsf{build successful} } \\
\end{minipage}
Convention to show a user command~: 
\begin{verbatim}
  make install
\end{verbatim}
Convention to show a file or a directory~: \\
\verb+  +\textit{lib/}\\
\verb+  +\textit{README.txt}

%\section{Problem reporting instructions}
%\begin{ndr}
%procedure de suivi de l'appli
%\end{ndr}


