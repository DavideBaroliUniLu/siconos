\section{Installation of the platform}

First of all, download the platform sources from CVS. Then, in order to install the external libraries needed by the platform, run in the root directory (\textit{SICONOS/})~:
\begin{verbatim}
  make instext
\end{verbatim}
And to compile the platform~:
\begin{verbatim}
  make all
\end{verbatim}

\section{Launch the platform}\label{run}
First of all, you have to set a \verb+SICONOSPATH+ environment variable to the complete path of the siconos root directory\footnote{for example "\$HOME/SICONOS" ; you can run "source siconos.csh"}. After, put your input files in the directory \textit{sample}. The input files are~:
\begin{itemize}
\item a command file in C++(like \textit{command.cpp}) which lead the execution\footnote{too see an example of command file report to Annexe A}~;
\item an \ac{xml} file which describes your system\footnote{it is optional. Anyway, you can put it in another directory}.
\item all the plugin you want to use in a C file (like \textit{myPlugin.c}).
\end{itemize}
Run the command~: 
\begin{verbatim}
  siconos yourCommandFile.cpp
\end{verbatim}
This will compile your plugins (if there are), your command file and link it to the platform.
In your command file, you can for example create a model using an \ac{xml} file place in the \textit{sample} directory.\\

Remark :
\begin{enumerate}
\item The siconos executable are in  \verb+$SICONOSPATH/siconos/+ directory.\\
\item If you want to use a plugin which is not in the directory \textit{sample} or in the directory \textit{src/plugin}, you must modify the \verb+LD_LIBRARY_PATH+ environment variable. It tell where to find libraries and plugins, so don't forget to add the path to this plugin in this environment variable (see \ref{plugin}).
\item The platform create a temporary file \textit{.siconos} in your directory. You have to be at least 20Mo of free space in your disk.  
\end{enumerate}


\section{Modify the platform}
If you have the source files of the platform, you can modify them.
After that, you must re-compile the platform with~:
\begin{verbatim}
  make all
\end{verbatim}

\section{Test the platform} 
If you modify the platform, you should run test to validate your modification. Do~:
\begin{verbatim}
  make buildtest
  make runtest
\end{verbatim}

This will launch a test suite with CppUnit.
If you add a class or a function, don't forget to make a test class.
After doing this, you can clean all the test data and binary file by~:
\begin{verbatim}
  make cleantest
\end{verbatim}

All commands can be launch on the root directory (build, run or clean all the test) or in a specified directory.

\section{Make a plugin}\label{plugin}

To make a plugin, create a file \textit{MyPlugin.c} and declare your functions with \verb+extern "C"+. Example for a vector field function~:
\begin{verbatim}
extern "C" void vectorField(int sizeOfX, double time, double& xPtr, double& xdotPtr)
{
    /* input parameter : sizeOfX (size of the vector X); 
     *                     time ; xPtr (pointer to X vector); 
     * output parameter : xdotPtr (pointer to Xdot vector)
     */

/*.....*/
}
\end{verbatim}
You must respect the parameter order for each function you declare. You can look at the \textit{BasicPlugin.c} file in \textit{src/plugin} directory to see examples.
To compile the plugin, do~:
\begin{verbatim}
  g++ -fPic -c myPlugin.c -o myPlugin.o
  g++ -fPic -shared -W -O myPlugin.o -o myPlugin.so
\end{verbatim}
The plugin \textit{myPlugin.so} is now created.
To use it, you have to precise its name and the function name in the \ac{xml} file (see xml part).
You can add the complete plugin path to the \verb+LD_LIBRARY_PATH+ environment variable. If you don't do this, you have to precise it in the \ac{siconos} \ac{xml} file.

Another way is to put the file \textit{MyPlugin.c} in the sample directory and run the siconos executable (see \ref{run}).
