% Siconos-Doc version 1.1.4, Copyright INRIA 2005-2006.
% Siconos is a program dedicated to modeling, simulation and control
% of non smooth dynamical systems.	
% Siconos is a free software; you can redistribute it and/or modify
% it under the terms of the GNU General Public License as published by
% the Free Software Foundation; either version 2 of the License, or
% (at your option) any later version.
% Siconos is distributed in the hope that it will be useful,
% but WITHOUT ANY WARRANTY; without even the implied warranty of
% MERCHANTABILITY or FITNESS FOR A PARTICULAR PURPOSE.  See the
% GNU General Public License for more details.
%
% You should have received a copy of the GNU General Public License
% along with Siconos; if not, write to the Free Software
% Foundation, Inc., 51 Franklin St, Fifth Floor, Boston, MA  02110-1301  USA
%
% Contact: Vincent ACARY vincent.acary@inrialpes.fr 
%/
\documentclass[10pt]{article}
%$Id: macro.tex,v 1.10 2004/12/08 13:38:58 acary Exp $


%\usepackage{a4wide}
\textheight 25cm
\textwidth 16.5cm
\topmargin -1cm
%\evensidemargin 0cm
\oddsidemargin 0cm
\evensidemargin0cm
\usepackage{layout}


\usepackage{amsmath}
\usepackage{amssymb}
\usepackage{minitoc}
%\usepackage{glosstex}
\usepackage{colortbl}
\usepackage{hhline}
\usepackage{longtable}

%\usepackage{glosstex}
%\def\glossaryname{Glossary of Notation}
\def\listacronymname{Acronyms}

\usepackage[outerbars]{changebar}\setcounter{changebargrey}{20}
%\glxitemorderdefault{acr}{l}

%\usepackage{color}
\usepackage{graphicx,epsfig}
\graphicspath{{figure/}}
\usepackage[T1]{fontenc}
\usepackage{rotating}

%\usepackage{algorithmic}
%\usepackage{algorithm}
\usepackage{ntheorem}
\usepackage{natbib}


%\renewcommand{\baselinestretch}{2.0}
\setcounter{tocdepth}{2}     % Dans la table des matieres
\setcounter{secnumdepth}{3}  % Avec un numero.



\newtheorem{definition}{Definition}
\newtheorem{lemma}{Lemma}
\newtheorem{claim}{Claim}
\newtheorem{remark}{Remark}
\newtheorem{assumption}{Assumption}
\newtheorem{example}{Example}
\newtheorem{conjecture}{Conjecture}
\newtheorem{corollary}{Corollary}
\newtheorem{OP}{OP}
\newtheorem{problem}{Problem}
\newtheorem{theorem}{Theorem}


\newcommand{\CC}{\mbox{\rm $~\vrule height6.6pt width0.5pt depth0.25pt\!\!$C}}
\newcommand{\ZZ}{\mbox{\rm \lower0.3pt\hbox{$\angle\!\!\!$}Z}}
\newcommand{\RR}{\mbox{\rm $I\!\!R$}}
\newcommand{\NN}{\mbox{\rm $I\!\!N$}}

\newcommand{\Mnn}{\mathcal M^{n\times n}}
\newcommand{\Mnp}[2]{\ensuremath{\mathcal M^{#1\times #2}}}



\newcommand{\Frac}[2]{\displaystyle \frac{#1}{#2}}

\newcommand{\DP}[2]{\displaystyle \frac{\partial {#1}}{\partial {#2}}}

% c++ variables writting
\newcommand{\varcpp}[1]{\textit{#1}}
% itemize
\newcommand{\bei}{\begin{itemize}}
\newcommand{\ei}{\end{itemize}}

\newcommand{\ie}{i.e.}
\newcommand{\eg}{e.g.}
\newcommand{\cf}{c.f.}
\newcommand{\putidx}[1]{\index{#1}\textit{#1}}

\def\Er{{\rm I\! R}}
\def\En{{\rm I\! N}} 
\def\Ec{{\rm I\! C}}
 
\def\zc{\hat{z}}
\def\wc{\hat{w}}

\font\tete=cmr8 at 8 pt
\font\titre= cmr12 at 20 pt 
\font\titregras=cmbx12 at 20 pt

%----------------------------------------------------------------------
%                  Modification des subsubsections
%----------------------------------------------------------------------
\makeatletter
\renewcommand\thesubsubsection{\thesubsection.\@alph\c@subsubsection}
\makeatother

%----------------------------------------------------------------------
%             Redaction note environnement
%----------------------------------------------------------------------
\makeatletter
\theoremheaderfont{\scshape}
\theoremstyle{marginbreak}
\theorembodyfont{\upshape}
%\newtheorem{rque}{\bf Remarque}[chapter]
%\newtheorem{rque1}{\bf \fsc{Remarque}}[chapter] !!! \fsc est une commande french
\newtheorem{ndr1}{\textbf{\textsc{Redaction note}}}[section]

\newenvironment{ndr}%
{%
\tt
%\centerline{---oOo---}
\noindent\begin{ndr1}%
}%
{%
\begin{flushright}%
%\vspace{-1.5em}\ding{111}
\end{flushright}%
\end{ndr1}%
%\centerline{---oOo---}
}

\makeatother

%----------------------------------------------------------------------
%             Redaction note environnement V.ACARY
%----------------------------------------------------------------------
\makeatletter
\theoremheaderfont{\scshape}
\theoremstyle{marginbreak}
\theorembodyfont{\upshape}
%\newtheorem{rque}{\bf Remarque}[chapter]
%\newtheorem{rque1}{\bf \fsc{Remarque}}[chapter] !!! \fsc est une commande french
\newtheorem{ndr1va}{\textbf{\textsc{Redaction note V. ACARY}}}[section]

\newenvironment{ndrva}%
{%
\tt
%\centerline{---oOo---}
\noindent\begin{ndr1va}%
}%
{%
\begin{flushright}%
%\vspace{-1.5em}\ding{111}
\end{flushright}%
\end{ndr1va}%
%\centerline{---oOo---}
}

\makeatother
%----------------------------------------------------------------------
%             Redaction note environnement V.ACARY
%----------------------------------------------------------------------
\makeatletter
\theoremheaderfont{\scshape}
\theoremstyle{marginbreak}
\theorembodyfont{\upshape}
%\newtheorem{rque}{\bf Remarque}[chapter]
%\newtheorem{rque1}{\bf \fsc{Remarque}}[chapter] !!! \fsc est une commande french
\newtheorem{ndr1fp}{\textbf{\textsc{Redaction note F. PERIGNON}}}[section]

\newenvironment{ndrfp}%
{%
\tt
%\centerline{---oOo---}
\noindent\begin{ndr1fp}%
}%
{%
\begin{flushright}%
%\vspace{-1.5em}\ding{111}
\end{flushright}%
\end{ndr1fp}%
%\centerline{---oOo---}
}

\makeatother
%----------------------------------------------------------------------
%                  Chapter head enviroment
%----------------------------------------------------------------------
\newenvironment{chapter_head}
{%
\begin{center}%
-------------------- oOo --------------------\\%
\ \\%
\begin{minipage}[]{14cm}%
\noindent\normalsize\advance\baselineskip-1pt %
}%
{%
\par\end{minipage}%
\ \\%
\ \\%
-------------------- oOo --------------------
\end{center}%
\vspace*{\stretch{1}}%
\clearpage%
\thispagestyle{empty}%
\vspace*{\stretch{1}}%
\minitoc%
\vspace*{\stretch{2}}%
\clearpage%
}

%%% Local Variables: 
%%% mode: latex
%%% TeX-master: "report"
%%% End: 

\usepackage{psfrag}
\usepackage{fancyhdr}
\usepackage{subfigure}
%\renewcommand{\baselinestretch}{1.2}
\textheight 23cm
\textwidth 16cm
\topmargin 0cm
%\evensidemargin 0cm
\oddsidemargin 0cm
\evensidemargin 0cm
\usepackage{layout}
\usepackage{mathpple}
\usepackage[T1]{fontenc}
%\usepackage{array}
\makeatletter
\renewcommand\bibsection{\paragraph{References
     \@mkboth{\MakeUppercase{\bibname}}{\MakeUppercase{\bibname}}}}
\makeatother
%% style des entetes et des pieds de page
\fancyhf{} % nettoie le entetes et les pieds
\fancyhead[L]{}
\fancyhead[R]{\thepage}
\fancyfoot[C]{}%
\begin{document}
\thispagestyle{empty}
\title{Short tutorial on STL tools and functions used in Siconos}
\author{F. P\'erignon}

\date{For Kernel version 1.1.4 \\
 \today}
\maketitle

\pagestyle{fancy}

\section{Introduction}
This paragraph is supposed to give to Siconos users a minimal knowledge on how to handle the STL functions and objects that are used in Siconos and thus necessary
to built properly a Non Smooth Dynamical System or a Strategy. \\
For more details on the STL tools, see for example \textit{http://www.sgi.com/tech/stl/table\_of\_contents.html} or \textit{ http://cplus.about.com/od/stl/ }. \\
Three main objects are widely used in Siconos:
\bei
\item map<key,object>: object that maps a ``key'' of any type to an ``object'' of any type. 
\item set<object>: a set of objects ...
\item vector<object>: a list of objects, with a specific sorting; data can also be accessed thanks to indices. Moreover, objects handled by a vector are contiguous in memory.
\ei

\section{Iterators and find function}\label{ItAndFind}

Iterators are a generalization of pointers: they are objects that point to other objects. They are used to iterate over a range of objects.
For example, if an iterator points to one element in a range, then it is possible to increment it so that it points to the next element. 

For example, suppose you have a set<DynamicalSystem*>, then you can define an iterator in the following way: \\
set<DynamicalSystem*>::iterator iter ; \\
iter will be used to access data in the set. \\

It is not necessary to give more details about iterators in that paragraph. Mainly, in Siconos users .cpp input files, they will only be used as a return value for find 
function. Then just used them as described in examples for map or set in the paragraphs below.

\section{Map handling}

Let A be a map<string,DynamicalSystem*>. \\

Data access: 
\bei
\item A[name] is the DynamicalSystem* that corresponds to the string name. Thus to fill a map in, just used A[name] = B, with B a DynamicalSystem*. 
\item A.erase(name) : (name of type string) erases the element whose key is name. 
\item A.clear(): removes all elements in the map. 
\item A.size() : number of elements in the map. 
\item A.find(name) or A.find(DS), name a string and DS a DynamicalSystem*: returns an iterator that points to DS. See example below.
\ei

Example:\\
\expandafter\ifx\csname indentation\endcsname\relax%
\newlength{\indentation}\fi
\setlength{\indentation}{0.5em}
\begin{flushleft}
\mbox{}\\
{$//$\it{} creates a map of DynamicalSystem$\ast${}\mbox{}\\
}\mbox{}\\
map$<$string, DynamicalSystem$\ast$$>$ A;\mbox{}\\
\mbox{}\\
DynamicalSystem $\ast$ DS = {\bf new} DynamicalSystem($\ldots$);\mbox{}\\
\mbox{}\\
{$//$\it{} Add some DynamicalSystem$\ast$ in the map{}\mbox{}\\
}A[{\tt"FirstDS"}] = DS;\mbox{}\\
A[{\tt"SecondDS"}] = {\bf new} DynamicalSystem($\ldots$);\mbox{}\\
$\ldots$\mbox{}\\
\mbox{}\\
map$<$string,DynamicalSystem$\ast$$>$::iterator iter;\mbox{}\\
\mbox{}\\
{$//$\it{} Find a DynamicalSystem$\ast$ in the map{}\mbox{}\\
}iter = A.find(DS);\mbox{}\\
\mbox{}\\
{$//$\it{} Then iter points to DS{}\mbox{}\\
}($\ast$iter)$\rightarrow$display() ; {$//$\it{} display data of DS{}\mbox{}\\
}\mbox{}\\
DynamicalSystem $\ast$ DS2 = {\bf new} DynamicalSystem($\ldots$);\mbox{}\\
iter = A.find(DS2);\mbox{}\\
\mbox{}\\
{$//$\it{} In that case, since DS2 is not in the map, {}\mbox{}\\
}{$//$\it{} iter is equal to A.end();{}\mbox{}\\
}{$//$\it{} It is then easy to test if a DynamicalSystem$\ast$ is in the map or not.{}\mbox{}\\
}\mbox{}\\
{\bf delete} DS;\mbox{}\\
{\bf delete} DS2;\mbox{}\\
{\bf delete} A[{\tt"secondDS"}];\mbox{}\\
A.clear();\mbox{}\\
\hspace*{1\indentation}\mbox{}\\
\end{flushleft}

\section{Set handling}

Let A<DynamicalSystem*> be a set. \\

Data access (DS being a DynamicalSystem*):
\bei
\item A.insert(DS) adds DS into the set
\item A.erase(DS)  removes DS from the set
\item A.find(DS) returns an iterator that points to DS. See example below.
\ei

Example:\\
\expandafter\ifx\csname indentation\endcsname\relax%
\newlength{\indentation}\fi
\setlength{\indentation}{0.5em}
\begin{flushleft}
{$//$\it{} creates a set of DynamicalSystem$\ast${}\mbox{}\\
}set$<$DynamicalSystem$\ast$$>$ A;\mbox{}\\
\mbox{}\\
DynamicalSystem$\ast$ DS = {\bf new} DynamicalSystem($\ldots$);\mbox{}\\
DynamicalSystem$\ast$ DS2 = {\bf new} DynamicalSystem($\ldots$);\mbox{}\\
\mbox{}\\
{$//$\it{} add elements into the set{}\mbox{}\\
}A.insert(DS);\mbox{}\\
A.insert(DS2);\mbox{}\\
\mbox{}\\
A.size(); {$//$\it{} is equal to 2. {}\mbox{}\\
}\mbox{}\\
{$//$\it{} find an element:{}\mbox{}\\
}set$<$DynamicalSystem$\ast$$>$::iterator iter;\mbox{}\\
iter = A.find(DS);\mbox{}\\
\mbox{}\\
{$//$\it{} then iter points to DS;{}\mbox{}\\
}($\ast$iter)$\rightarrow$display(); {$//$\it{} display DS data. {}\mbox{}\\
}\mbox{}\\
{$//$\it{} remove an element{}\mbox{}\\
}A.remove(DS2); \mbox{}\\
{$//$\it{} A.size() is then equal to 1.{}\mbox{}\\
}\mbox{}\\
iter = A.find(DS2);\mbox{}\\
{$//$\it{} then iter $=$ A.end(), which means that DS2 is not in the set anymore.{}\mbox{}\\
}\mbox{}\\
{\bf delete} DS;\mbox{}\\
{\bf delete} DS2;\mbox{}\\
\end{flushleft}

\section{Vectors handling}

Let V<DynamicalSystem*> be a vector. \\

Data access: 
\bei
\item V[i] is the component at position i in the vector, and so is a DynamicalSystem*. 
\item V.size(): number of elements in V. 
\item V.push\_back(DS) : adds the DynamicalSystem* DS at the end of V. 
\item V.pop\_back() : removes the last element of V. 
\item V.clear() : removes all elements. 
\ei

Examples: \\
\expandafter\ifx\csname indentation\endcsname\relax%
\newlength{\indentation}\fi
\setlength{\indentation}{0.5em}
\begin{flushleft}
{$//$\it{} Creates a vector of three elements, that contains DynamicalSystem$\ast${}\mbox{}\\
}vector$<$DynamicalSystem$\ast$$>$ V(3);\mbox{}\\
V[0] = {\bf new} DynamicalSystem($\ldots$);\mbox{}\\
\mbox{}\\
DynamicalSystem$\ast$ DS = {\bf new} DynamicalSystem($\ldots$);\mbox{}\\
V[1] = DS;\mbox{}\\
\mbox{}\\
V[2] = NULL; \mbox{}\\
\mbox{}\\
{$//$\it{} At this points, V.size() is equal to 3.{}\mbox{}\\
}\mbox{}\\
DynamicalSystem$\ast$ DS2 = {\bf new} DynamicalSystem($\ldots$);\mbox{}\\
\mbox{}\\
V.push\_back(DS2);\mbox{}\\
\mbox{}\\
{$//$\it{} then V.size() is equal to 4.{}\mbox{}\\
}\mbox{}\\
{$//$\it{} DS display: {}\mbox{}\\
}V[1]$\rightarrow$display();\mbox{}\\
\mbox{}\\
$\ldots$\mbox{}\\
\mbox{}\\
{\bf delete} DS2;\mbox{}\\
{\bf delete} DS;\mbox{}\\
{\bf delete} V[0];\mbox{}\\
V.clear();\mbox{}\\
\end{flushleft}

\end{document}
