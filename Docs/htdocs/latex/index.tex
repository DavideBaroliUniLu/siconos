  \hyperlink{index_about}{About} \hyperlink{index_news}{News} \hyperlink{index_contacts}{Contacts} \hyperlink{index_links}{Useful Links}\par
   Last stable release is v2.0.0. \par
 \href{https://gforge.inria.fr/frs/?group_id=9}{\tt Download Siconos } \par
 Installation, tutorial, mailing lists ...: \par
 \hyperlink{SiconosDocList}{Siconos Help and Documentation.}\par
      \hypertarget{index_about}{}\section{About}\label{index_about}
The Siconos Platform is a scientific computing software dedicated to the modeling, simulation, control and analysis of {\bf Non Smooth Dynamical Systems (NSDS)}. This project belongs to the European project \href{http://siconos.inrialpes.fr/}{\tt SICONOS}, dedicated to the modeling, the control the analysis and the simulation of the so-called \char`\"{}Non Smooth Dynamical system\char`\"{}.\par
 \hypertarget{index_MainObj}{}\subsection{What are the main objectives and motivations?}\label{index_MainObj}
Besides the standard features which are required for a software of scientific computing, the objectives of this project are of the following :\begin{itemize}
\item To provide a common framework (modeling and simulation tools) for non smooth problems present in various scientific fields : applied Mathematics, Mechanics, Robotics, Electrical networks, etc.\item To be able to rely on existing developments, as well for the modeling tools as for the simulation tools. The platform will not re-implement the dedicated tools, which are already used for the modeling of specific systems in various fields,but will provide a framework to their integration. In the same way, we want to re-use improved numerical low-level routines.\item To support exchanges and comparisons of methods between researchers.\item To disseminate the know-how to other fields of research and industry.\item To take into account the diversity of users (end users, experts, software developers, framework builders, industrial users, etc ... ).\item To set up standards in terms of modeling of the NSDS. -To ensure software quality by the use of modern software design methods. \end{itemize}
\hypertarget{index_license}{}\subsection{License}\label{index_license}
This software is distributed under the \href{http://www.gnu.org/copyleft/gpl.html}{\tt GNU General Public License} \hypertarget{index_publi}{}\subsection{Related publications}\label{index_publi}
You can find here some publications at the origin of this software on \href{http://siconos.inrialpes.fr/}{\tt Siconos European Project Home Page} \hypertarget{index_news}{}\section{News}\label{index_news}
\begin{itemize}
\item {\em  September 7th 2006\/}: release 2.0.0 available on \href{http://gforge.inria.fr/projects/siconos}{\tt Gforge} (include {\bf Event} Driven algorithm)\item {\em  June 5th 2006:\/} release 1.2.0\item {\em  November 30th 2005:\/} second release.\item {\em  October 10th 2005:\/} first release is available. \end{itemize}
\hypertarget{index_contacts}{}\section{Contacts}\label{index_contacts}
\hypertarget{index_Mailing}{}\subsection{lists}\label{index_Mailing}
See Mailing lists at \href{https://gforge.inria.fr/mail/?group_id=9}{\tt Gforge.} \hypertarget{index_forum}{}\subsection{Forum}\label{index_forum}
See forum list at \href{https://gforge.inria.fr/forum/?group_id=9}{\tt Gforge.} \hypertarget{index_bug}{}\subsection{Bug reports}\label{index_bug}
To report a bug, or see previously reported bugs and fixes, please use our bug tracker on the \href{https://gforge.inria.fr/tracker/?group_id=9}{\tt Gforge} \hypertarget{index_gencontact}{}\subsection{General contacts}\label{index_gencontact}
\href{mailto:vincent.acary@ nospam inrialpes.fr}{\tt Vincent Acary}\par
 \href{mailto:franck.perignon@ nospam inrialpes.fr}{\tt Franck P\'{e}rignon}\par
\hypertarget{index_links}{}\section{Useful Links}\label{index_links}
\hyperlink{developments}{Developments tools and libraries} \par
 \hyperlink{related}{Related projects and Platforms} \par
 \hyperlink{scientificComputing}{Scientific computing}

{\em  Pages generated by \href{mailto:franck.perignon@ nospam inrialpes.fr}{\tt Franck P\'{e}rignon} \/} 